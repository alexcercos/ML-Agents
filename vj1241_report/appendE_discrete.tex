% ==============================================================================
% This file is part of the "LaTeX template for writing the Final Degree Work
% report". It has been developed to aid the students of the Bachelor's Degree in
% Video Game Design and Development at the Jaume I University.
%
% (c) 2019 Sergio Barrachina Mir and Jos� Vte. Mart� Avil�s
%
% The template can be used and distributed under the next license:
%  Creative Commons Attribution-NonCommercial-ShareAlike (CC BY-NC-SA)
%  http://creativecommons.org/licenses/by-nc-sa/3.0/
%  http://creativecommons.org/licenses/by-nc-sa/3.0/legalcode
%
% Atom editor configuration follows:
% !TEX root = ./report.tex
% !TeX spellcheck = en-US
% ==============================================================================
\addtocontents{toc}{\protect\setcounter{tocdepth}{0}}

\chapter{Discrete actions}
\label{app:discrete}

This appendix contains some notes that were taken when (and after) training sessions related to modeling discrete (binary) actions.

Any parameter that does not appear in one training is set to default (see ./config/trainer.config file).

Trained models with green titles are considered good or any improvement in the investigation. Models with red titles are considered failures.

Some of the models have not been saved, either because they don't perform well or they perform in much the same way as another model (agent).

Most of the notes should not be taken literally or as certain, as they usually are theories or preliminary conclusions drawn during the training itself.

\section*{\color{red} 17/6 17:57 (6\textunderscore 17\textunderscore 1757)}

Trainer: PPO

\begin{center}
	\begin{tabular}{ | m{4cm} | m{2.5cm}||m{4cm} | m{2.5cm} | } 
		\hline
			batch size & 512 & beta & 5.0e-3 \\ 
		\hline
			buffer size & 2048 & epsilon & 0.2 \\
		\hline
		  hidden units & 256 & learning rate & 7.5e-4\\
		\hline
			learning rate schedule & linear & max steps & 6.0e5\\
		\hline
			memory size & 32 & normalize & false\\
		\hline
			num epoch & 4 & num layers & 3 \\
		\hline
			sequence length & 128 & summary freq & 1000 \\
		\hline
			time horizon & 64 & use recurrent & false\\
		\hline
			extrinsic strength & 1.0 & \textbf{extrinsic gamma} & \textbf{0.99}\\
		\hline
			curiosity strength & 0.02 & curiosity gamma & 0.99\\
		\hline
			curiosity encoding size & 128 & & \\
		\hline
	\end{tabular}
\end{center}

\textbf{Render Target Sensor: 64x64}

Grayscale: false

\textbf{Observation space size = 20}

Action space size = 1

\vspace{2mm}

Episode steps = 1000

Total steps = 294000

Time = 6800s

\vspace{2mm}

Click training (test) with cheat sheet: TR=0.1, pun and rew=10 
For some reason, the bot doesn't click on compiled trainings, but it does in the editor.

Some incorrect impulses disappear over time, but the rewards seem very unbalanced.

\section*{\color{green} 17/6 19:54 (6\textunderscore 17\textunderscore 1954)}

Trainer: PPO

\begin{center}
	\begin{tabular}{ | m{4cm} | m{2.5cm}||m{4cm} | m{2.5cm} | } 
		\hline
			batch size & 512 & beta & 5.0e-3 \\ 
		\hline
			buffer size & 2048 & epsilon & 0.2 \\
		\hline
		  hidden units & 256 & learning rate & 7.5e-4\\
		\hline
			learning rate schedule & linear & max steps & 6.0e5\\
		\hline
			memory size & 32 & normalize & false\\
		\hline
			num epoch & 4 & num layers & 3 \\
		\hline
			sequence length & 128 & summary freq & 1000 \\
		\hline
			time horizon & 64 & use recurrent & false\\
		\hline
			extrinsic strength & 1.0 & extrinsic gamma & 0.99\\
		\hline
			curiosity strength & 0.02 & curiosity gamma & 0.99\\
		\hline
			curiosity encoding size & 128 & & \\
		\hline
	\end{tabular}
\end{center}

Render Target Sensor: 64x64

Grayscale: false

Observation space size = 20

Action space size = 1

\vspace{2mm}

Episode steps = 1000

Total steps = 600000

Time = 13800s

\vspace{2mm}

15 of punish factor, 100 of reward factor and 0.2 of tolerable range. It improves quite a bit but has no penalty for leaving clicks, so it doesn't hit all of them (and gets a high reward).

\section*{\color{red} 18/6 11:00 (6\textunderscore 18\textunderscore 1100)}

Trainer: PPO

\begin{center}
	\begin{tabular}{ | m{4cm} | m{2.5cm}||m{4cm} | m{2.5cm} | } 
		\hline
			batch size & 512 & beta & 5.0e-3 \\ 
		\hline
			buffer size & 2048 & epsilon & 0.2 \\
		\hline
		  hidden units & 256 & learning rate & 7.5e-4\\
		\hline
			learning rate schedule & linear & max steps & 6.0e5\\
		\hline
			memory size & 32 & normalize & false\\
		\hline
			num epoch & 4 & num layers & 3 \\
		\hline
			sequence length & 128 & summary freq & 1000 \\
		\hline
			time horizon & 64 & use recurrent & false\\
		\hline
			extrinsic strength & 1.0 & extrinsic gamma & 0.99\\
		\hline
			curiosity strength & 0.02 & curiosity gamma & 0.99\\
		\hline
			curiosity encoding size & 128 & & \\
		\hline
	\end{tabular}
\end{center}

Render Target Sensor: 64x64

Grayscale: false

Observation space size = 20

Action space size = 1

\vspace{2mm}

Episode steps = 1000

Total steps = 570000

Time = 12200s

\vspace{2mm}

Training with penalties for failing impulses (-4 * punFactor) added. I don't like very much that it relies on the interpolation variable to see the accuracy because in very short movements it's very difficult to get it right. In next trainings try with SAC or BC, and then think about improving the reward system.

I think the penalty was too high, now it makes too many impulses. Next time I try SAC and less penalty.

\section*{\color{green} 18/6 14:26 (6\textunderscore 18\textunderscore 1426)}

Trainer: SAC

\begin{center}
	\begin{tabular}{ | m{4cm} | m{2.5cm}||m{4cm} | m{2.5cm} | } 
		\hline
			batch size & 128 & buffer size & 200000 \\ 
		\hline
			buffer init steps & 5000 & hidden units & 256 \\
		\hline
			init entcoef & 1.0 & learning rate & 5.0e-4\\
		\hline
			learning rate schedule & constant & max steps & 6.0e6\\
		\hline
			memory size & 32 & normalize & true\\
		\hline
			num update & 1 & train interval & 5\\
		\hline
			\textbf{num layers} & \textbf{2} & time horizon & 64\\
		\hline
			sequence length & 128 & summary freq & 1000\\
		\hline
			tau & 0.005 & use recurrent & false \\
		\hline
			vis encode type & simple & & \\
		\hline
			extrinsic strength & 1.0 & extrinsic gamma & 0.99\\
		\hline
			curiosity strength & 0.1 & curiosity gamma & 0.99\\
		\hline
			curiosity encoding size & 128 & & \\
		\hline
	\end{tabular}
\end{center}

Render Target Sensor: 64x64

Grayscale: false

Observation space size = 20

Action space size = 1

\vspace{2mm}

Episode steps = 1000

Total steps = 488000

Time = 17300s

\vspace{2mm}

Training with SAC, 2*punish for leaving impulses and the same parameters. I have also reduced the amount of layers (although maybe one would be enough).
It hits quite a few impulses, except for very specific (and maybe confusing) cases. It reaches the rewards of 6\textunderscore 17\textunderscore 1954, but with even stricter conditions.

\section*{\color{red} 19/6 10:45 (6\textunderscore 19\textunderscore 1045)}

Trainer: BC

\begin{center}
	\begin{tabular}{ | m{4cm} | m{2.5cm}||m{4cm} | m{2.5cm} | } 
		\hline
			batches per epoch & 10 & batch size & 64 \\ 
		\hline
			hidden units & 128 & learning rate & 1.0e-5 \\ 
		\hline
			max steps & 5.0e4 & memory size & 32 \\ 
		\hline
			num layers & 4 & sequence length & 128 \\ 
		\hline
			summary freq & 1000 & use recurrent & false \\ 
		\hline
	\end{tabular}
\end{center}

Demo path: demos/Clicker09.demo

\vspace{2mm}

Render Target Sensor: 64x64

Grayscale: false

Observation space size = 20

Action space size = 1

\vspace{2mm}

Episode steps = 1000

Total steps = 50000

Time = 5200s

\vspace{2mm}

I had several ValueErrors because the heuristics created a size 2 action space (not compatible with the real one). 

The model, although it gets some impulses right, is very imprecise. It doesn't seem to improve the SAC or PPO trainings at all.

\section*{\color{green} 19/6 13:11 (6\textunderscore 19\textunderscore 1311)}

Trainer: SAC

\begin{center}
	\begin{tabular}{ | m{4cm} | m{2.5cm}||m{4cm} | m{2.5cm} | } 
		\hline
			batch size & 128 & buffer size & 200000 \\ 
		\hline
			buffer init steps & 5000 & hidden units & 256 \\
		\hline
			init entcoef & 1.0 & learning rate & 5.0e-4\\
		\hline
			learning rate schedule & constant & max steps & 6.0e6\\
		\hline
			memory size & 32 & normalize & true\\
		\hline
			num update & 1 & train interval & 5\\
		\hline
			num layers & 2 & time horizon & 64\\
		\hline
			sequence length & 128 & summary freq & 1000\\
		\hline
			tau & 0.005 & use recurrent & false \\
		\hline
			vis encode type & simple & & \\
		\hline
			extrinsic strength & 1.0 & extrinsic gamma & 0.99\\
		\hline
			curiosity strength & 0.1 & curiosity gamma & 0.99\\
		\hline
			curiosity encoding size & 128 & & \\
		\hline
	\end{tabular}
\end{center}

Render Target Sensor: 64x64

Grayscale: false

Observation space size = 20

Action space size = 1

\vspace{2mm}

Episode steps = 1000

Total steps = 348000

Time = 12800s

\vspace{2mm}

Tolerable range reduced to 0.1. The reward should be lower than the previous model with SAC, but it will be necessary to check if it is better in practice.

It seems to work well. Comparison with previous model with range 0.2 is missing.

\section*{\color{red} 19/6 19:59 (6\textunderscore 19\textunderscore 1959)}

Trainer: SAC

\begin{center}
	\begin{tabular}{ | m{4cm} | m{2.5cm}||m{4cm} | m{2.5cm} | } 
		\hline
			batch size & 128 & buffer size & 200000 \\ 
		\hline
			buffer init steps & 5000 & hidden units & 256 \\
		\hline
			init entcoef & 1.0 & learning rate & 5.0e-4\\
		\hline
			learning rate schedule & constant & max steps & 6.0e6\\
		\hline
			memory size & 32 & normalize & true\\
		\hline
			num update & 1 & train interval & 5\\
		\hline
			num layers & 2 & time horizon & 64\\
		\hline
			sequence length & 128 & summary freq & 1000\\
		\hline
			tau & 0.005 & use recurrent & false \\
		\hline
			vis encode type & simple & & \\
		\hline
			extrinsic strength & 1.0 & extrinsic gamma & 0.99\\
		\hline
			curiosity strength & 0.1 & curiosity gamma & 0.99\\
		\hline
			curiosity encoding size & 128 & & \\
		\hline
	\end{tabular}
\end{center}

Render Target Sensor: 64x64

Grayscale: false

Observation space size = 20

Action space size = 1

\vspace{2mm}

Episode steps = 1000

Total steps = 452000

Time = 16000s

\vspace{2mm}

Noise factor of 2 added (calculated differently from other noises). However, I can't get the movement to be stable in Y (it goes in both directions).

At 390k steps it diverges. Then it seems that the graph stays flat.

\section*{\color{red} 22/6 12:13 (6\textunderscore 22\textunderscore 1213)}

Trainer: SAC

\begin{center}
	\begin{tabular}{ | m{4cm} | m{2.5cm}||m{4cm} | m{2.5cm} | } 
		\hline
			batch size & 128 & buffer size & 200000 \\ 
		\hline
			buffer init steps & 5000 & hidden units & 256 \\
		\hline
			init entcoef & 1.0 & learning rate & 5.0e-4\\
		\hline
			learning rate schedule & constant & max steps & 6.0e6\\
		\hline
			memory size & 32 & normalize & true\\
		\hline
			num update & 1 & train interval & 5\\
		\hline
			num layers & 2 & time horizon & 64\\
		\hline
			sequence length & 128 & summary freq & 1000\\
		\hline
			tau & 0.005 & use recurrent & false \\
		\hline
			vis encode type & simple & & \\
		\hline
			extrinsic strength & 1.0 & extrinsic gamma & 0.99\\
		\hline
			curiosity strength & 0.1 & curiosity gamma & 0.99\\
		\hline
			curiosity encoding size & 128 & & \\
		\hline
	\end{tabular}
\end{center}

Render Target Sensor: 64x64

Grayscale: false

Observation space size = 20

Action space size = 1

\vspace{2mm}

Episode steps = 1000

Total steps = -

Time = -

\vspace{2mm}

New bot with multiple inaccuracies (BotPrecision): inaccuracy=1.2, noise in X = 0.8, noise in Y = 0.1, timeClic (average) = 0.9, time click variance = 0.1. Targets are now also detected while in idle. The penalties are exaggeratedly large (around 10,000), so I should balance them out in subsequent trainings.

\section*{\color{green} 22/6 12:43 (6\textunderscore 22\textunderscore 1243)}

Trainer: SAC

\begin{center}
	\begin{tabular}{ | m{4cm} | m{2.5cm}||m{4cm} | m{2.5cm} | } 
		\hline
			batch size & 128 & buffer size & 200000 \\ 
		\hline
			buffer init steps & 5000 & hidden units & 256 \\
		\hline
			init entcoef & 1.0 & learning rate & 5.0e-4\\
		\hline
			learning rate schedule & constant & max steps & 6.0e6\\
		\hline
			memory size & 32 & normalize & true\\
		\hline
			num update & 1 & train interval & 5\\
		\hline
			num layers & 2 & time horizon & 64\\
		\hline
			sequence length & 128 & summary freq & 1000\\
		\hline
			tau & 0.005 & use recurrent & false \\
		\hline
			vis encode type & simple & & \\
		\hline
			extrinsic strength & 1.0 & extrinsic gamma & 0.99\\
		\hline
			curiosity strength & 0.1 & curiosity gamma & 0.99\\
		\hline
			curiosity encoding size & 128 & & \\
		\hline
	\end{tabular}
\end{center}

Render Target Sensor: 64x64

Grayscale: false

Observation space size = 20

Action space size = 1

\vspace{2mm}

Episode steps = 1000

Total steps = 126000

Time = 4500s

\vspace{2mm}

The error was that the agent was getting punishments from the bot that didn't move instead of the new one (it has to be changed in the function of the rewards, since it's a getComponent).

The result is decent as almost any of the previous ones (even if this one has more noise), but sometimes the bot clicks without coming to mind.

\section*{\color{green} 22/6 14:02 (6\textunderscore 22\textunderscore 1402)}

Trainer: SAC

\begin{center}
	\begin{tabular}{ | m{4cm} | m{2.5cm}||m{4cm} | m{2.5cm} | } 
		\hline
			batch size & 128 & buffer size & 200000 \\ 
		\hline
			buffer init steps & 5000 & hidden units & 256 \\
		\hline
			init entcoef & 1.0 & learning rate & 5.0e-4\\
		\hline
			learning rate schedule & constant & max steps & 6.0e6\\
		\hline
			memory size & 32 & normalize & true\\
		\hline
			num update & 1 & train interval & 5\\
		\hline
			num layers & 2 & time horizon & 64\\
		\hline
			sequence length & 128 & summary freq & 1000\\
		\hline
			tau & 0.005 & use recurrent & false \\
		\hline
			\textbf{vis encode type} & \textbf{resnet} & & \\
		\hline
			extrinsic strength & 1.0 & extrinsic gamma & 0.99\\
		\hline
			curiosity strength & 0.1 & curiosity gamma & 0.99\\
		\hline
			curiosity encoding size & 128 & & \\
		\hline
	\end{tabular}
\end{center}

Render Target Sensor: 64x64

Grayscale: false

Observation space size = 20

Action space size = 1

\vspace{2mm}

Episode steps = 1000

Total steps = 118000

Time = 6200s

\vspace{2mm}

Use of resnet as a visual encoder. It is expected to be slower, but it should also work better.

I think it always acts in front of the target as it destroys it, but it has the potential to act (it does it in the middle). It could end the training being the network that makes the clicks instead of the bot (if I can get the bot not to go crazy by not hitting the target).

I've activated the neural network for a few steps, but it has the problem that if it's the network that determines when to click, the agent's shots are not shown (I don't know if it penalizes or not).
It works like the previous ones, in the next trainings I will change to half agent control (working well).

The performance of the game when using the net is as good as with other visual encoders (the training is a bit slower), but the result seems to be a bit better.

It causes failures when it doesn't click when playing the neural network, the bot stands pointing at a target but doesn't move. That problem is tried to be solved in the next training.

\section*{\color{red} 22/6 16:03 (6\textunderscore 22\textunderscore 1603)}

Trainer: SAC

\begin{center}
	\begin{tabular}{ | m{4cm} | m{2.5cm}||m{4cm} | m{2.5cm} | } 
		\hline
			batch size & 128 & buffer size & 200000 \\ 
		\hline
			buffer init steps & 5000 & hidden units & 256 \\
		\hline
			init entcoef & 1.0 & learning rate & 5.0e-4\\
		\hline
			learning rate schedule & constant & max steps & 6.0e6\\
		\hline
			memory size & 32 & normalize & true\\
		\hline
			num update & 1 & train interval & 5\\
		\hline
			num layers & 2 & time horizon & 64\\
		\hline
			sequence length & 128 & summary freq & 1000\\
		\hline
			tau & 0.005 & use recurrent & false \\
		\hline
			vis encode type & resnet & & \\
		\hline
			extrinsic strength & 1.0 & extrinsic gamma & 0.99\\
		\hline
			curiosity strength & 0.1 & curiosity gamma & 0.99\\
		\hline
			curiosity encoding size & 128 & & \\
		\hline
	\end{tabular}
\end{center}

Render Target Sensor: 64x64

Grayscale: false

Observation space size = 20

Action space size = 1

\vspace{2mm}

Episode steps = 1000

Total steps = 65000

Time = 3600s

\vspace{2mm}

Fixed some bugs where the clicks were not displayed as the runtime was programmed. Now the agent can be trained as the main player (although it's not recommended to do it from the beginning).

Switch to neural network control at 40,000 steps.

Be careful to avoid getting stuck on the site if it doesn't respond: penalize strongly and continue after X failures.

The problem of switching is to get stuck if it doesn't respond, maybe if it is trained from the beginning with agent control it will have better results.

\section*{\color{red} 22/6 17:25 (6\textunderscore 22\textunderscore 1725)}

Trainer: SAC

\begin{center}
	\begin{tabular}{ | m{4cm} | m{2.5cm}||m{4cm} | m{2.5cm} | } 
		\hline
			batch size & 128 & buffer size & 200000 \\ 
		\hline
			buffer init steps & 5000 & hidden units & 256 \\
		\hline
			init entcoef & 1.0 & learning rate & 5.0e-4\\
		\hline
			learning rate schedule & constant & max steps & 6.0e6\\
		\hline
			memory size & 32 & normalize & true\\
		\hline
			num update & 1 & train interval & 5\\
		\hline
			num layers & 2 & time horizon & 64\\
		\hline
			sequence length & 128 & summary freq & 1000\\
		\hline
			tau & 0.005 & use recurrent & false \\
		\hline
			vis encode type & resnet & & \\
		\hline
			extrinsic strength & 1.0 & extrinsic gamma & 0.99\\
		\hline
			curiosity strength & 0.1 & curiosity gamma & 0.99\\
		\hline
			curiosity encoding size & 128 & & \\
		\hline
	\end{tabular}
\end{center}

Render Target Sensor: 64x64

Grayscale: false

Observation space size = 20

Action space size = 1

\vspace{2mm}

Episode steps = 1000

Total steps = 16000

Time = 700s

\vspace{2mm}

Train the bot from the beginning, the number of simultaneous planes has been limited again. Increased penalty for failing to click to 3 times punFactor (now it has worse consequences to leave a click than to fail it).

After several changes it stays at *5 for not clicking and *2 for clicking in idle or too late.

Try PPO in next trainings.

Maybe I could put as an input parameter the last value of the network (to know if it has clicked or not).

\section*{\color{green} 22/6 17:40 (6\textunderscore 22\textunderscore 1740)}

Trainer: SAC

\begin{center}
	\begin{tabular}{ | m{4cm} | m{2.5cm}||m{4cm} | m{2.5cm} | } 
		\hline
			batch size & 128 & buffer size & 200000 \\ 
		\hline
			buffer init steps & 5000 & hidden units & 256 \\
		\hline
			init entcoef & 1.0 & learning rate & 5.0e-4\\
		\hline
			learning rate schedule & constant & max steps & 6.0e6\\
		\hline
			memory size & 32 & normalize & true\\
		\hline
			num update & 1 & train interval & 5\\
		\hline
			num layers & 2 & time horizon & 64\\
		\hline
			sequence length & 128 & summary freq & 1000\\
		\hline
			tau & 0.005 & use recurrent & false \\
		\hline
			vis encode type & resnet & & \\
		\hline
			extrinsic strength & 1.0 & extrinsic gamma & 0.99\\
		\hline
			curiosity strength & 0.1 & curiosity gamma & 0.99\\
		\hline
			curiosity encoding size & 128 & & \\
		\hline
	\end{tabular}
\end{center}

Render Target Sensor: 64x64

Grayscale: false

\textbf{Observation space size = 21}

Action space size = 1

\vspace{2mm}

Episode steps = 1000

Total steps = 71000

Time = 3700s

\vspace{2mm}

Added the last click value as a parameter (if it's higher than 0, it should force the bot to go negative so the next frame can be clicked). I think that in previous trainings it got stuck because when it received only the last image, it considered that it should stay in positive values.

Clicking too much should give a small penalty (sometimes it makes 3 clicks in a row, but it gets 0 points). The result is still much better.

\section*{\color{green} 22/6 18:47 (6\textunderscore 22\textunderscore 1847)}

Trainer: SAC

\begin{center}
	\begin{tabular}{ | m{4cm} | m{2.5cm}||m{4cm} | m{2.5cm} | } 
		\hline
			batch size & 128 & buffer size & 200000 \\ 
		\hline
			buffer init steps & 5000 & hidden units & 256 \\
		\hline
			init entcoef & 1.0 & learning rate & 5.0e-4\\
		\hline
			learning rate schedule & constant & max steps & 6.0e6\\
		\hline
			memory size & 32 & normalize & true\\
		\hline
			num update & 1 & train interval & 5\\
		\hline
			num layers & 2 & time horizon & 64\\
		\hline
			sequence length & 128 & summary freq & 1000\\
		\hline
			tau & 0.005 & use recurrent & false \\
		\hline
			vis encode type & resnet & & \\
		\hline
			extrinsic strength & 1.0 & extrinsic gamma & 0.99\\
		\hline
			curiosity strength & 0.1 & curiosity gamma & 0.99\\
		\hline
			curiosity encoding size & 128 & & \\
		\hline
	\end{tabular}
\end{center}

Render Target Sensor: 64x64

Grayscale: false

Observation space size = 21

Action space size = 1

\vspace{2mm}

Episode steps = 1000

Total steps = 245000

Time = 6500s

\vspace{2mm}

PunFactor/3 punishment has been added for over-clicking (it may not have much effect, since the bot has no click memory, or it may force it to perform better).

It has had a crash for unknown reasons (it doesn't show up in the log editor either). However, the reward already seemed to be stagnant for a long time. The model works well.

\section*{\color{red} 23/6 10:13 (6\textunderscore 23\textunderscore 1013)}

Trainer: PPO

\begin{center}
	\begin{tabular}{ | m{4cm} | m{2.5cm}||m{4cm} | m{2.5cm} | } 
		\hline
			batch size & 512 & beta & 5.0e-3 \\ 
		\hline
			buffer size & 2048 & epsilon & 0.2 \\
		\hline
		  hidden units & 256 & learning rate & 7.5e-4\\
		\hline
			learning rate schedule & linear & max steps & 6.0e5\\
		\hline
			memory size & 32 & normalize & false\\
		\hline
			num epoch & 4 & num layers & 3 \\
		\hline
			sequence length & 128 & summary freq & 1000 \\
		\hline
			time horizon & 64 & use recurrent & false\\
		\hline
			extrinsic strength & 1.0 & extrinsic gamma & 0.99\\
		\hline
			curiosity strength & 0.02 & curiosity gamma & 0.99\\
		\hline
			curiosity encoding size & 128 & & \\
		\hline
	\end{tabular}
\end{center}

Render Target Sensor: 64x64

Grayscale: false

Observation space size = 21

Action space size = 1

\vspace{2mm}

Episode steps = 1000

Total steps = 33000

Time = 770s

\vspace{2mm}

Training with DISCRETE space of action (1 branch of size 2). The rest of the training works the same way. In this case it has a simple encoder, I stop the training to compare it with resnet.

\section*{\color{red} 23/6 10:28 (6\textunderscore 23\textunderscore 1028)}

Trainer: PPO

\begin{center}
	\begin{tabular}{ | m{4cm} | m{2.5cm}||m{4cm} | m{2.5cm} | } 
		\hline
			batch size & 512 & beta & 5.0e-3 \\ 
		\hline
			buffer size & 2048 & epsilon & 0.2 \\
		\hline
		  hidden units & 256 & learning rate & 7.5e-4\\
		\hline
			learning rate schedule & linear & max steps & 6.0e5\\
		\hline
			memory size & 32 & normalize & false\\
		\hline
			num epoch & 4 & num layers & 3 \\
		\hline
			sequence length & 128 & summary freq & 1000 \\
		\hline
			time horizon & 64 & use recurrent & false\\
		\hline
			\textbf{vis encode type} & \textbf{resnet} & &\\
		\hline
			extrinsic strength & 1.0 & extrinsic gamma & 0.99\\
		\hline
			curiosity strength & 0.02 & curiosity gamma & 0.99\\
		\hline
			curiosity encoding size & 128 & & \\
		\hline
	\end{tabular}
\end{center}

Render Target Sensor: 64x64

Grayscale: false

Observation space size = 21

Action space size = 1

\vspace{2mm}

Episode steps = 1000

Total steps = 109000

Time = 2600s

\vspace{2mm}

Change on the encoder to compare performance. Use SAC in discrete space for the next workout.

The reward curve increases but after a number of steps it stays flat (in the graph the agent is stuck at 1, but without clicking).

\section*{\color{red} 23/6 11:18 (6\textunderscore 23\textunderscore 1118)}

Trainer: SAC

\begin{center}
	\begin{tabular}{ | m{4cm} | m{2.5cm}||m{4cm} | m{2.5cm} | } 
		\hline
			batch size & 128 & buffer size & 200000 \\ 
		\hline
			buffer init steps & 5000 & hidden units & 256 \\
		\hline
			init entcoef & 1.0 & learning rate & 5.0e-4\\
		\hline
			learning rate schedule & constant & max steps & 6.0e6\\
		\hline
			memory size & 32 & normalize & true\\
		\hline
			num update & 1 & train interval & 5\\
		\hline
			num layers & 2 & time horizon & 64\\
		\hline
			sequence length & 128 & summary freq & 1000\\
		\hline
			tau & 0.005 & use recurrent & false \\
		\hline
			vis encode type & resnet & & \\
		\hline
			extrinsic strength & 1.0 & extrinsic gamma & 0.99\\
		\hline
			curiosity strength & 0.1 & curiosity gamma & 0.99\\
		\hline
			curiosity encoding size & 128 & & \\
		\hline
	\end{tabular}
\end{center}

Render Target Sensor: 64x64

Grayscale: false

Observation space size = 21

Action space size = 1

\vspace{2mm}

Episode steps = 1000

Total steps = 54000

Time = 2750s

\vspace{2mm}

Training with discrete action space and SAC. Before 10000 steps the reward already falls and the agent stagnates and does not improve again.

\section*{\color{red} 23/6 12:06 (6\textunderscore 23\textunderscore 1206)}

Trainer: SAC

\begin{center}
	\begin{tabular}{ | m{4cm} | m{2.5cm}||m{4cm} | m{2.5cm} | } 
		\hline
			batch size & 128 & buffer size & 200000 \\ 
		\hline
			buffer init steps & 5000 & hidden units & 256 \\
		\hline
			init entcoef & 1.0 & learning rate & 5.0e-4\\
		\hline
			learning rate schedule & constant & max steps & 6.0e6\\
		\hline
			memory size & 32 & normalize & true\\
		\hline
			num update & 1 & train interval & 5\\
		\hline
			num layers & 2 & time horizon & 64\\
		\hline
			sequence length & 128 & summary freq & 1000\\
		\hline
			tau & 0.005 & use recurrent & false \\
		\hline
			vis encode type & resnet & & \\
		\hline
			extrinsic strength & 1.0 & extrinsic gamma & 0.99\\
		\hline
			curiosity strength & 0.1 & curiosity gamma & 0.99\\
		\hline
			curiosity encoding size & 128 & & \\
		\hline
	\end{tabular}
\end{center}

Render Target Sensor: 64x64

Grayscale: false

Observation space size = 21

Action space size = 1

\vspace{2mm}

Episode steps = 1000

Total steps = 38000

Time = 1800s

\vspace{2mm}

Same training but lastClick is saved as a random number (instead of 0 or 1). Same result.

\section*{\color{red} 23/6 13:05 (6\textunderscore 23\textunderscore 1305)}

Trainer: SAC

\begin{center}
	\begin{tabular}{ | m{4cm} | m{2.5cm}||m{4cm} | m{2.5cm} | } 
		\hline
			batch size & 128 & buffer size & 200000 \\ 
		\hline
			buffer init steps & 5000 & hidden units & 256 \\
		\hline
			init entcoef & 1.0 & learning rate & 5.0e-4\\
		\hline
			learning rate schedule & constant & max steps & 6.0e6\\
		\hline
			memory size & 32 & normalize & true\\
		\hline
			num update & 1 & train interval & 5\\
		\hline
			num layers & 2 & time horizon & 64\\
		\hline
			sequence length & 128 & summary freq & 1000\\
		\hline
			tau & 0.005 & use recurrent & false \\
		\hline
			vis encode type & resnet & & \\
		\hline
			extrinsic strength & 1.0 & extrinsic gamma & 0.99\\
		\hline
			curiosity strength & 0.1 & curiosity gamma & 0.99\\
		\hline
			curiosity encoding size & 128 & & \\
		\hline
	\end{tabular}
\end{center}

Render Target Sensor: 64x64

Grayscale: false

Observation space size = 21

Action space size = 1

\vspace{2mm}

Episode steps = 1000

Total steps = 13000

Time = 600s

\vspace{2mm}

Training with continuous values again. A factor has been added whereby movements with short reaction times discount less and give more reward. I don't know how the model can get better or worse with these values. It has learned but somewhat worse, it's not valid because of a failure in the calculation of the new factors.

\section*{\color{green} 23/6 13:17 (6\textunderscore 23\textunderscore 1317)}

Trainer: SAC

\begin{center}
	\begin{tabular}{ | m{4cm} | m{2.5cm}||m{4cm} | m{2.5cm} | } 
		\hline
			batch size & 128 & buffer size & 200000 \\ 
		\hline
			buffer init steps & 5000 & hidden units & 256 \\
		\hline
			init entcoef & 1.0 & learning rate & 5.0e-4\\
		\hline
			learning rate schedule & constant & max steps & 6.0e6\\
		\hline
			memory size & 32 & normalize & true\\
		\hline
			num update & 1 & train interval & 5\\
		\hline
			num layers & 2 & time horizon & 64\\
		\hline
			sequence length & 128 & summary freq & 1000\\
		\hline
			tau & 0.005 & use recurrent & false \\
		\hline
			vis encode type & resnet & & \\
		\hline
			extrinsic strength & 1.0 & extrinsic gamma & 0.99\\
		\hline
			curiosity strength & 0.1 & curiosity gamma & 0.99\\
		\hline
			curiosity encoding size & 128 & & \\
		\hline
	\end{tabular}
\end{center}

Render Target Sensor: 64x64

Grayscale: false

Observation space size = 21

Action space size = 1

\vspace{2mm}

Episode steps = 1000

Total steps = 91000

Time = 4600s

\vspace{2mm}

Error corrected. Correct results are obtained, but they are not very different in quality from those that did not take time into account. However, there are times when it acts differently (it stays up while it doesn't have to click, and in a while it goes back down).

\section*{\color{red} 23/6 18:55 (6\textunderscore 23\textunderscore 1855)}

Trainer: SAC

\begin{center}
	\begin{tabular}{ | m{4cm} | m{2.5cm}||m{4cm} | m{2.5cm} | } 
		\hline
			batch size & 128 & buffer size & 200000 \\ 
		\hline
			buffer init steps & 5000 & hidden units & 256 \\
		\hline
			init entcoef & 1.0 & learning rate & 5.0e-4\\
		\hline
			learning rate schedule & constant & max steps & 6.0e6\\
		\hline
			memory size & 32 & normalize & true\\
		\hline
			num update & 1 & train interval & 5\\
		\hline
			num layers & 2 & time horizon & 64\\
		\hline
			sequence length & 128 & summary freq & 1000\\
		\hline
			tau & 0.005 & use recurrent & false \\
		\hline
			vis encode type & resnet & & \\
		\hline
			extrinsic strength & 1.0 & extrinsic gamma & 0.99\\
		\hline
			curiosity strength & 0.1 & curiosity gamma & 0.99\\
		\hline
			curiosity encoding size & 128 & & \\
		\hline
	\end{tabular}
\end{center}

Render Target Sensor: 64x64

Grayscale: false

Observation space size = 21

Action space size = 1

\vspace{2mm}

Episode steps = 1000

Total steps = 7000

Time = 250s

\vspace{2mm}

Training with discrete action space but considering that it clicks whenever iClick is 1 (in each frame independently, without depending on whether the previous one was 0).

The initial rewards are very low due to the large amount of clicks it makes, I hope it won't diverge because of that.

In a matter of 5000 steps it has already learned not to make clicks, since not acting is more profitable than constantly failing.

\section*{\color{green} 23/6 19:05 (6\textunderscore 23\textunderscore 1905)}

Trainer: SAC

\begin{center}
	\begin{tabular}{ | m{4cm} | m{2.5cm}||m{4cm} | m{2.5cm} | } 
		\hline
			batch size & 128 & buffer size & 200000 \\ 
		\hline
			buffer init steps & 5000 & hidden units & 256 \\
		\hline
			init entcoef & 1.0 & learning rate & 5.0e-4\\
		\hline
			learning rate schedule & constant & max steps & 6.0e6\\
		\hline
			memory size & 32 & normalize & true\\
		\hline
			num update & 1 & train interval & 5\\
		\hline
			num layers & 2 & time horizon & 64\\
		\hline
			sequence length & 128 & summary freq & 1000\\
		\hline
			tau & 0.005 & use recurrent & false \\
		\hline
			vis encode type & resnet & & \\
		\hline
			extrinsic strength & 1.0 & extrinsic gamma & 0.99\\
		\hline
			curiosity strength & 0.1 & curiosity gamma & 0.99\\
		\hline
			curiosity encoding size & 128 & & \\
		\hline
	\end{tabular}
\end{center}

Render Target Sensor: 64x64

Grayscale: false

Observation space size = 21

Action space size = 1

\vspace{2mm}

Episode steps = 1000

Total steps = 66000

Time = 2800s

\vspace{2mm}

PunFactor of 2, RewFactor of 30, and ignoring a click is penalized *20.
It works well but not better than its continuous motion counterparts, it has more uncounted clicks than the other models (although it usually gets it right when it should).

\section*{\color{red} 24/6 13:06 (6\textunderscore 24\textunderscore 1306)}

Trainer: BC

\begin{center}
	\begin{tabular}{ | m{4cm} | m{2.5cm}||m{4cm} | m{2.5cm} | } 
		\hline
			batches per epoch & 10 & batch size & 64 \\ 
		\hline
			hidden units & 128 & learning rate & 1.0e-5 \\ 
		\hline
			max steps & 5.0e4 & memory size & 32 \\ 
		\hline
			num layers & 4 & sequence length & 128 \\ 
		\hline
			summary freq & 1000 & use recurrent & false \\ 
		\hline
	\end{tabular}
\end{center}

\textbf{Demo path: demos/ClickerDiscrete.demo}

\vspace{2mm}

Render Target Sensor: 64x64

Grayscale: false

Observation space size = 21

Action space size = 1

\vspace{2mm}

Episode steps = 1000

Total steps = 22000

Time = 2680s

\vspace{2mm}

Behavioral cloning with discrete movements. It has the problem of learning to act in the next frame (it should be commented in a section) because the demos are 1 frame late, when the target has already been destroyed.

