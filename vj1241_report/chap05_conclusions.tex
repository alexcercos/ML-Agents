% ==============================================================================
% This file is part of the "LaTeX template for writing the Final Degree Work
% report". It has been developed to aid the students of the Bachelor's Degree in
% Video Game Design and Development at the Jaume I University.
%
% (c) 2019 Sergio Barrachina Mir and José Vte. Martí Avilés
%
% The template can be used and distributed under the next license:
%  Creative Commons Attribution-NonCommercial-ShareAlike (CC BY-NC-SA)
%  http://creativecommons.org/licenses/by-nc-sa/3.0/
%  http://creativecommons.org/licenses/by-nc-sa/3.0/legalcode
%
% Atom editor configuration follows:
% !TEX root = ./report.tex
% !TeX spellcheck = en-US
% ==============================================================================

\chapter{Conclusions and Future Work}

\minitoc{}

\bigskip{}

In this chapter, the conclusions of the work, as well as its future extensions are shown.

\section{Conclusions}

As preliminary conclusions, it has been proved that SAC policy is better than PPO to solve our task since it requires smaller datasets (or steps) and develops better behaviors than PPO (even though, PPO is better for testing since it is faster).

ML Agents makes easier the development of neural network and its inclusion in 3D environments, however it still has some errors that complicate this process. Moreover, its policies are not completely optimized for GPU usage, and some simple convolutions slow the training process heavily. Also, the latest versions of CUDA and tensorflow are not supported.

To end with, being able to design a custom neural network could improve the imitation results. It may be possible to do with ML Agents, but it could also be dangerous since it is necessary to modify source code in python. However, at this point we have already reached some of the milestones initially proposed by modeling reactive movements properly.

\section{Future work}

This framework could be continued in more complex games, using real player data to model its behaviors. However, I don't plan on doing it in the near future since the computing needed would be very high and it should be trained outside of Unity (and incorporated into real games).