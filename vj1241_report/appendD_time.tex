% ==============================================================================
% This file is part of the "LaTeX template for writing the Final Degree Work
% report". It has been developed to aid the students of the Bachelor's Degree in
% Video Game Design and Development at the Jaume I University.
%
% (c) 2019 Sergio Barrachina Mir and Jos� Vte. Mart� Avil�s
%
% The template can be used and distributed under the next license:
%  Creative Commons Attribution-NonCommercial-ShareAlike (CC BY-NC-SA)
%  http://creativecommons.org/licenses/by-nc-sa/3.0/
%  http://creativecommons.org/licenses/by-nc-sa/3.0/legalcode
%
% Atom editor configuration follows:
% !TEX root = ./report.tex
% !TeX spellcheck = en-US
% ==============================================================================
\addtocontents{toc}{\protect\setcounter{tocdepth}{0}}

\chapter{Reaction time}
\label{app:time}

This appendix contains some notes that were taken when (and after) training sessions related to modeling delayed reactive behaviors, or in other words, agents with reaction time.

Any parameter that does not appear in one training is set to default (see ./config/trainer.config file).

Trained models with green titles are considered good or any improvement in the investigation. Models with red titles are considered failures.

Some of the models have not been saved, either because they don't perform well or they perform in much the same way as another model (agent).

Most of the notes should not be taken literally or as certain, as they usually are theories or preliminary conclusions drawn during the training itself.

\section*{\color{green} 09/5 18:55 (5\textunderscore 09\textunderscore 1855)}

Trainer: PPO

\begin{center}
	\begin{tabular}{ | m{4cm} | m{2.5cm}||m{4cm} | m{2.5cm} | } 
		\hline
			batch size & 1024 & beta & 5.0e-3 \\ 
		\hline
			buffer size & 8192 & epsilon & 0.2 \\
		\hline
		  hidden units & 256 & learning rate & 2.0e-3\\
		\hline
			learning rate schedule & linear & max steps & 6.0e5\\
		\hline
			normalize & true & num epoch & 6\\
		\hline
			num layers & 1 & summary freq & 1000\\
		\hline
			time horizon & 64 & &\\
		\hline
			pretraining strength & 0.8 & pretraining steps & 20000\\
		\hline
			extrinsic strength & 1.0 & extrinsic gamma & 0.99\\
		\hline
			curiosity strength & 0.01 & curiosity gamma & 0.99\\
		\hline
			curiosity encoding size & 128 & & \\
		\hline
			gail strength & 0.01 & gail gamma & 0.99\\
		\hline
			gail encoding size & 128 & & \\
		\hline
	\end{tabular}
\end{center}

Demo path: demos/IndDemo.demo

\vspace{2mm}

Render Target Sensor: 64x64

Grayscale: false

Observation space size = 0

Action space size = 2

\vspace{2mm}

Episode steps = 1000

Total steps = 524000

Time = 5580s

\vspace{2mm}

Speed test using ppo with 3 envs and GPU, same rewards as in 5\textunderscore 09\textunderscore 1643

It works twice as fast, so it can be good for testing (and using SAC for final versions). However, it often gives errors like Out of memory (and other stranger ones that may be caused by overheating).

\section*{\color{green} 12/5 20:00 (5\textunderscore 12\textunderscore 2000)}

Trainer: SAC

\begin{center}
	\begin{tabular}{ | m{4cm} | m{2.5cm}||m{4cm} | m{2.5cm} | } 
		\hline
			batch size & 128 & buffer size & 200000 \\ 
		\hline
			buffer init steps & 5000 & hidden units & 256 \\
		\hline
			init entcoef & 1.0 & learning rate & 4.0e-4\\
		\hline
			learning rate schedule & constant & max steps & 6.0e5\\
		\hline
			memory size & 256 & normalize & true\\
		\hline
			num update & 1 & train interval & 5\\
		\hline
			num layers & 1 & time horizon & 64\\
		\hline
			sequence length & 128 & summary freq & 1000\\
		\hline
			tau & 0.005 & use recurrent & false\\
		\hline
			vis encode type & simple & & \\
		\hline
			extrinsic strength & 1.5 & extrinsic gamma & 0.99\\
		\hline
			curiosity strength & 0.03 & curiosity gamma & 0.99\\
		\hline
			curiosity encoding size & 128 & & \\
		\hline
	\end{tabular}
\end{center}

\textbf{Frames~\footnote{Frame 0 is the current frame, which is used in every training, the other frames are the n-th previous frame. }: 0, 5, 10, 12, 15, 17, 20, 22, 25, 27, 30, 32, 35, 37, 40, 45, 50, 55, 60, 70, 80}

\vspace{2mm}

\textbf{Render Target Sensor: 21 * (20x20)}

\textbf{Grayscale: true}

Observation space size = 0

Action space size = 2

\vspace{2mm}

Episode steps = 1000

Total steps = 278000

Time = 17650s

\vspace{2mm}

