% ==============================================================================
% This file is part of the "LaTeX template for writing the Final Degree Work
% report". It has been developed to aid the students of the Bachelor's Degree in
% Video Game Design and Development at the Jaume I University.
%
% (c) 2019 Sergio Barrachina Mir and José Vte. Martí Avilés
%
% The template can be used and distributed under the next license:
%  Creative Commons Attribution-NonCommercial-ShareAlike (CC BY-NC-SA)
%  http://creativecommons.org/licenses/by-nc-sa/3.0/
%  http://creativecommons.org/licenses/by-nc-sa/3.0/legalcode
%
% Atom editor configuration follows:
% !TEX root = ./report.tex
% !TeX spellcheck = en-US
% ==============================================================================

\chapter{Dynamic average}

\label{app:average}

Starting from the average formula:
$ {\displaystyle \frac1n\sum_{i=1}^{n}x_i=a_n} $\\\\

\noindent
We solve the equation for $a_{n+1}$:

\noindent
$ {\displaystyle \sum_{i=1}^{n}x_i=n\cdot a_n} $\\\\
\noindent
$ {\displaystyle \sum_{i=1}^{n}x_i-n\cdot a_n=\sum_{i=1}^{n+1}x_i-(n+1)\cdot a_{n+1}=0} $\\\\
\noindent
$ {\displaystyle \sum_{i=1}^{n}x_i-n\cdot a_n=\sum_{i=1}^{n}x_i+x_{n+1}-(n+1)\cdot a_{n+1}} $\\\\
\noindent
$ {\displaystyle (n+1)\cdot a_{n+1}=x_{n+1}+n\cdot a_n} $\\\\
\noindent
$ {\displaystyle a_{n+1}=\frac{x_{n+1}+n\cdot a_n}{n+1}} $\\\\

\vspace{5mm}

Then, it has been proven that the average of N+1 elements can be obtained with cost O(1) knowing the average of N elements and the new element.
