% ==============================================================================
% This file is part of the "LaTeX template for writing the Final Degree Work
% report". It has been developed to aid the students of the Bachelor's Degree in
% Video Game Design and Development at the Jaume I University.
%
% (c) 2019 Sergio Barrachina Mir and Jos� Vte. Mart� Avil�s
%
% The template can be used and distributed under the next license:
%  Creative Commons Attribution-NonCommercial-ShareAlike (CC BY-NC-SA)
%  http://creativecommons.org/licenses/by-nc-sa/3.0/
%  http://creativecommons.org/licenses/by-nc-sa/3.0/legalcode
%
% Atom editor configuration follows:
% !TEX root = ./report.tex
% !TeX spellcheck = en-US
% ==============================================================================
\addtocontents{toc}{\protect\setcounter{tocdepth}{0}}

\chapter{Complex movements}
\label{app:complex}

This appendix contains some notes that were taken when (and after) training sessions related to modeling more complex movements.

Any parameter that does not appear in one training is set to default (see ./config/trainer.config file).

Trained models with green titles are considered good or any improvement in the investigation. Models with red titles are considered failures.

Some of the models have not been saved, either because they don't perform well or they perform in much the same way as another model (agent).

Most of the notes should not be taken literally or as certain, as they usually are theories or preliminary conclusions drawn during the training itself.

\section*{\color{red} 25/6 11:34 (6\textunderscore 25\textunderscore 1134)}

Trainer: PPO

\begin{center}
	\begin{tabular}{ | m{4cm} | m{2.5cm}||m{4cm} | m{2.5cm} | } 
		\hline
			batch size & 512 & beta & 5.0e-3 \\ 
		\hline
			buffer size & 2048 & epsilon & 0.2 \\
		\hline
		  hidden units & 256 & learning rate & 7.5e-4\\
		\hline
			learning rate schedule & linear & max steps & 6.0e5\\
		\hline
			memory size & 32 & normalize & false\\
		\hline
			num epoch & 4 & num layers & 3 \\
		\hline
			sequence length & 128 & summary freq & 1000 \\
		\hline
			time horizon & 64 & use recurrent & false\\
		\hline
			vis encode type & resnet & &\\
		\hline
			extrinsic strength & 1.0 & extrinsic gamma & 0.99\\
		\hline
			curiosity strength & 0.02 & curiosity gamma & 0.99\\
		\hline
			curiosity encoding size & 128 & & \\
		\hline
	\end{tabular}
\end{center}

Render Target Sensor: 64x64

Grayscale: false

\textbf{Observation space size = 20}

\textbf{Action space size = 2}

\vspace{2mm}

Episode steps = 1000

Total steps = 82000

Time = 2700s

\vspace{2mm}

Noiseless and accurate bot (and movement curve with increasing speed), rewards with tolerable range of 0.15 and factors of 30 (reward and penalty). 

At the beginning it shows a cuda out of memory error that in the last ones didn't show up, I don't know if it could be because the bot is bigger, because of the encoder or because of using the 2 debugs at the same time.

At the end it has failed (GPU sync error). The learning has followed very strange patterns, it always gave values ending in 11 and it has been jumping between 3 concrete rewards: 7211, 7811, 8411 (and once 9011). 

\section*{\color{red} 25/6 12:46 (6\textunderscore 25\textunderscore 1246)}

Trainer: PPO

\begin{center}
	\begin{tabular}{ | m{4cm} | m{2.5cm}||m{4cm} | m{2.5cm} | } 
		\hline
			batch size & 512 & beta & 5.0e-3 \\ 
		\hline
			buffer size & 2048 & epsilon & 0.2 \\
		\hline
		  hidden units & 256 & learning rate & 7.5e-4\\
		\hline
			learning rate schedule & linear & max steps & 6.0e5\\
		\hline
			memory size & 32 & normalize & false\\
		\hline
			num epoch & 4 & num layers & 3 \\
		\hline
			sequence length & 128 & summary freq & 1000 \\
		\hline
			time horizon & 64 & use recurrent & false\\
		\hline
			vis encode type & resnet & &\\
		\hline
			extrinsic strength & 1.0 & extrinsic gamma & 0.99\\
		\hline
			curiosity strength & 0.02 & curiosity gamma & 0.99\\
		\hline
			curiosity encoding size & 128 & & \\
		\hline
	\end{tabular}
\end{center}

Render Target Sensor: 64x64

Grayscale: false

Observation space size = 20

Action space size = 2

\vspace{2mm}

Episode steps = 1000

Total steps = 20000

Time = 700s

\vspace{2mm}

The distance function doesn't seem to be going wrong, now punFactor=10, rewFactor=30 and maxRange=2 (of tolerable range). Out of memory errors still appear.

For some reason it follows the same patterns as before, even with the same times. It fails again as before.

\section*{\color{red} 25/6 13:02 (6\textunderscore 25\textunderscore 1302)}

Trainer: PPO

\begin{center}
	\begin{tabular}{ | m{4cm} | m{2.5cm}||m{4cm} | m{2.5cm} | } 
		\hline
			batch size & 512 & beta & 5.0e-3 \\ 
		\hline
			buffer size & 2048 & epsilon & 0.2 \\
		\hline
		  hidden units & 256 & learning rate & 7.5e-4\\
		\hline
			learning rate schedule & linear & max steps & 6.0e5\\
		\hline
			memory size & 32 & normalize & false\\
		\hline
			num epoch & 4 & num layers & 3 \\
		\hline
			sequence length & 128 & summary freq & 1000 \\
		\hline
			time horizon & 64 & use recurrent & false\\
		\hline
			\textbf{vis encode type} & \textbf{simple} & &\\
		\hline
			extrinsic strength & 1.0 & extrinsic gamma & 0.99\\
		\hline
			curiosity strength & 0.02 & curiosity gamma & 0.99\\
		\hline
			curiosity encoding size & 128 & & \\
		\hline
	\end{tabular}
\end{center}

Render Target Sensor: 64x64

Grayscale: false

Observation space size = 20

Action space size = 2

\vspace{2mm}

Episode steps = 1000

Total steps = 29000

Time = 650s

\vspace{2mm}

Change in the encoder. No out of memory errors at first. In the next one I'm going to put resnet again, but with a network with less internal layers.

\section*{\color{red} 25/6 13:22 (6\textunderscore 25\textunderscore 1322)}

Trainer: PPO

\begin{center}
	\begin{tabular}{ | m{4cm} | m{2.5cm}||m{4cm} | m{2.5cm} | } 
		\hline
			batch size & 512 & beta & 5.0e-3 \\ 
		\hline
			buffer size & 2048 & epsilon & 0.2 \\
		\hline
		  hidden units & 256 & learning rate & 7.5e-4\\
		\hline
			learning rate schedule & linear & max steps & 6.0e5\\
		\hline
			memory size & 32 & normalize & false\\
		\hline
			num epoch & 4 & \textbf{num layers} & \textbf{2} \\
		\hline
			sequence length & 128 & summary freq & 1000 \\
		\hline
			\textbf{time horizon} & \textbf{32} & use recurrent & false\\
		\hline
			\textbf{vis encode type} & \textbf{resnet} & &\\
		\hline
			extrinsic strength & 1.0 & extrinsic gamma & 0.99\\
		\hline
			curiosity strength & 0.02 & curiosity gamma & 0.99\\
		\hline
			curiosity encoding size & 128 & & \\
		\hline
	\end{tabular}
\end{center}

\textbf{Render Target Sensor: 32x32}

Grayscale: false

Observation space size = 20

Action space size = 2

\vspace{2mm}

Episode steps = 1000

Total steps = 49000

Time = 1000s

\vspace{2mm}

The error was that the cheat rewards missed were added (this always happened).

Training with the same parameters but simpler (and with less time horizon, which could decrease even more in theory).

Image size has been decreased to see if the problem is there (resnet should need then less memory).

It seems recommended to decrease the batch (and buffer) size to improve the performance, I will try it in next trainings.

For some reason, the debug axis lines are "destroyed" (the object disappears) and that causes an error in Unity, we'll have to see where the bug is.

On the other hand, the rewards are very small (even being multiplied by 30, but at the end you divide everything by 1000).

In the end it started to rise, but the axis debug was destroyed.
The mistake was that the debug's sphere had a collider, and when the bot shot at it, the parent object (the whole axis debug) was destroyed. This shouldn't be a problem now.

\section*{\color{red} 25/6 14:03 (6\textunderscore 25\textunderscore 1403)}

Trainer: PPO

\begin{center}
	\begin{tabular}{ | m{4cm} | m{2.5cm}||m{4cm} | m{2.5cm} | } 
		\hline
			batch size & 512 & beta & 5.0e-3 \\ 
		\hline
			buffer size & 2048 & epsilon & 0.2 \\
		\hline
		  hidden units & 256 & learning rate & 7.5e-4\\
		\hline
			learning rate schedule & linear & max steps & 6.0e5\\
		\hline
			memory size & 32 & normalize & false\\
		\hline
			num epoch & 4 & num layers & 2 \\
		\hline
			sequence length & 128 & summary freq & 1000 \\
		\hline
			time horizon & 32 & use recurrent & false\\
		\hline
			vis encode type & resnet & &\\
		\hline
			extrinsic strength & 1.0 & extrinsic gamma & 0.99\\
		\hline
			curiosity strength & 0.02 & curiosity gamma & 0.99\\
		\hline
			curiosity encoding size & 128 & & \\
		\hline
	\end{tabular}
\end{center}

Render Target Sensor: 32x32

Grayscale: false

Observation space size = 20

Action space size = 2

\vspace{2mm}

Episode steps = 1000

Total steps = 291000

Time = 5900s

\vspace{2mm}

Fixed bugs and Rew-Pun factors multiplied by 10 (100-300). 

It goes up slowly but steadily, it may be more optimal to force the bot to have less reward (with the same factors, as before). In the graph it seems to be tending to the same side but with a lot of noise, in this case it would also be convenient to reduce the tolerable range.

\section*{\color{red} 25/6 15:46 (6\textunderscore 25\textunderscore 1546)}

Trainer: PPO

\begin{center}
	\begin{tabular}{ | m{4cm} | m{2.5cm}||m{4cm} | m{2.5cm} | } 
		\hline
			batch size & 512 & beta & 5.0e-3 \\ 
		\hline
			buffer size & 2048 & epsilon & 0.2 \\
		\hline
		  hidden units & 256 & learning rate & 7.5e-4\\
		\hline
			learning rate schedule & linear & max steps & 6.0e5\\
		\hline
			memory size & 32 & normalize & false\\
		\hline
			num epoch & 4 & num layers & 2 \\
		\hline
			sequence length & 128 & summary freq & 1000 \\
		\hline
			time horizon & 32 & use recurrent & false\\
		\hline
			vis encode type & resnet & &\\
		\hline
			extrinsic strength & 1.0 & extrinsic gamma & 0.99\\
		\hline
			curiosity strength & 0.02 & curiosity gamma & 0.99\\
		\hline
			curiosity encoding size & 128 & & \\
		\hline
	\end{tabular}
\end{center}

Render Target Sensor: 32x32

Grayscale: false

Observation space size = 20

Action space size = 2

\vspace{2mm}

Episode steps = 1000

Total steps = 480000

Time = 11200s

\vspace{2mm}

RewFactor=PunFactor=300, TR=0.1. It does improve the reward but not so much the behavior. Check how it goes with 2 debug axes.

\section*{\color{red} 25/6 18:59 (6\textunderscore 25\textunderscore 1859)}

Trainer: SAC

\begin{center}
	\begin{tabular}{ | m{4cm} | m{2.5cm}||m{4cm} | m{2.5cm} | } 
		\hline
			batch size & 128 & buffer size & 200000 \\ 
		\hline
			buffer init steps & 5000 & hidden units & 256 \\
		\hline
			init entcoef & 1.0 & learning rate & 5.0e-4\\
		\hline
			learning rate schedule & constant & max steps & 6.0e6\\
		\hline
			memory size & 32 & normalize & true\\
		\hline
			num update & 1 & train interval & 5\\
		\hline
			num layers & 2 & time horizon & 64\\
		\hline
			sequence length & 128 & summary freq & 1000\\
		\hline
			tau & 0.005 & use recurrent & false \\
		\hline
			vis encode type & resnet & & \\
		\hline
			extrinsic strength & 1.0 & extrinsic gamma & 0.99\\
		\hline
			curiosity strength & 0.1 & curiosity gamma & 0.99\\
		\hline
			curiosity encoding size & 128 & & \\
		\hline
	\end{tabular}
\end{center}

Render Target Sensor: 32x32

Grayscale: false

Observation space size = 20

Action space size = 2

\vspace{2mm}

Episode steps = 1000

Total steps = 248000

Time = 8900s

\vspace{2mm}

It ends up converging to a kind of medium value, but it achieves positive rewards. The result is not very good also because the observations were of the agent (at first almost random, and then very uniform). In the next one reduce the tolerable range and pass pure observations of the bot.

\section*{\color{green} 25/6 21:32 (6\textunderscore 25\textunderscore 2132)}

Trainer: SAC

\begin{center}
	\begin{tabular}{ | m{4cm} | m{2.5cm}||m{4cm} | m{2.5cm} | } 
		\hline
			batch size & 128 & buffer size & 200000 \\ 
		\hline
			buffer init steps & 5000 & hidden units & 256 \\
		\hline
			init entcoef & 1.0 & learning rate & 5.0e-4\\
		\hline
			learning rate schedule & constant & max steps & 6.0e6\\
		\hline
			memory size & 32 & normalize & true\\
		\hline
			num update & 1 & train interval & 5\\
		\hline
			num layers & 2 & time horizon & 64\\
		\hline
			sequence length & 128 & summary freq & 1000\\
		\hline
			tau & 0.005 & use recurrent & false \\
		\hline
			vis encode type & resnet & & \\
		\hline
			extrinsic strength & 1.0 & extrinsic gamma & 0.99\\
		\hline
			curiosity strength & 0.1 & curiosity gamma & 0.99\\
		\hline
			curiosity encoding size & 128 & & \\
		\hline
	\end{tabular}
\end{center}

Render Target Sensor: 32x32

Grayscale: false

Observation space size = 20

Action space size = 2

\vspace{2mm}

Episode steps = 1000

Total steps = 270000

Time = 9700s

\vspace{2mm}

0.05 of TR and observations of the bot. It has managed to adapt well in a moment, but over 250,000 steps it has started to become unstable. The noise in Y is much higher than the noise in X.

It is considered good because it has been close to a good result (although it does not work well in the editor, since it used bot observations).

\section*{\color{red} 26/6 11:59 (6\textunderscore 26\textunderscore 1159)}

Trainer: SAC

\begin{center}
	\begin{tabular}{ | m{4cm} | m{2.5cm}||m{4cm} | m{2.5cm} | } 
		\hline
			batch size & 128 & buffer size & 200000 \\ 
		\hline
			buffer init steps & 5000 & hidden units & 256 \\
		\hline
			init entcoef & 1.0 & learning rate & 5.0e-4\\
		\hline
			learning rate schedule & constant & max steps & 6.0e6\\
		\hline
			memory size & 32 & normalize & true\\
		\hline
			num update & 1 & train interval & 5\\
		\hline
			num layers & 2 & time horizon & 64\\
		\hline
			sequence length & 128 & summary freq & 1000\\
		\hline
			tau & 0.005 & use recurrent & false \\
		\hline
			vis encode type & resnet & & \\
		\hline
			extrinsic strength & 1.0 & extrinsic gamma & 0.99\\
		\hline
			curiosity strength & 0.1 & curiosity gamma & 0.99\\
		\hline
			curiosity encoding size & 128 & & \\
		\hline
	\end{tabular}
\end{center}

Render Target Sensor: 32x32

Grayscale: false

Observation space size = 20

Action space size = 2

\vspace{2mm}

Episode steps = 1000

Total steps = 88000

Time = 3100s

\vspace{2mm}

Training with angle-magnitude tolerable range rewards TRAngle = 0.05, TRMagnitude = 0.05, MR = 0.5, punFactor=200, rewFactor=300. 

I think there's some failure in the rewards because it's moving almost the opposite way and getting less negative rewards each time, I'll let it go to see how it goes.

\section*{\color{green} 26/6 13:29 (6\textunderscore 26\textunderscore 1329)}

Trainer: SAC

\begin{center}
	\begin{tabular}{ | m{4cm} | m{2.5cm}||m{4cm} | m{2.5cm} | } 
		\hline
			batch size & 128 & buffer size & 200000 \\ 
		\hline
			buffer init steps & 5000 & hidden units & 256 \\
		\hline
			init entcoef & 1.0 & learning rate & 5.0e-4\\
		\hline
			learning rate schedule & constant & max steps & 6.0e6\\
		\hline
			memory size & 32 & normalize & true\\
		\hline
			num update & 1 & train interval & 5\\
		\hline
			num layers & 2 & time horizon & 64\\
		\hline
			sequence length & 128 & summary freq & 1000\\
		\hline
			tau & 0.005 & use recurrent & false \\
		\hline
			vis encode type & resnet & & \\
		\hline
			extrinsic strength & 1.0 & extrinsic gamma & 0.99\\
		\hline
			curiosity strength & 0.1 & curiosity gamma & 0.99\\
		\hline
			curiosity encoding size & 128 & & \\
		\hline
	\end{tabular}
\end{center}

Render Target Sensor: 32x32

Grayscale: false

Observation space size = 20

Action space size = 2

\vspace{2mm}

Episode steps = 1000

Total steps = 260000

Time = 9000s

\vspace{2mm}

The problem was in how the arc tangents were applied, so that the opposite one gave the same result as the positive one (-X/-Y==X/Y). Now I use Vector2.Angle to calculate it directly (in decimal).

The final result isn't bad at all but it is quite noisy and slow (and still not out of the bot's observations). I should add sensitivity to the movement in the Y axis (and maybe in the X axis) because the bot has more trouble adjusting to very small numbers.

\section*{\color{green} 26/6 16:12 (6\textunderscore 26\textunderscore 1612)}

Trainer: SAC

\begin{center}
	\begin{tabular}{ | m{4cm} | m{2.5cm}||m{4cm} | m{2.5cm} | } 
		\hline
			batch size & 128 & buffer size & 200000 \\ 
		\hline
			buffer init steps & 5000 & hidden units & 256 \\
		\hline
			init entcoef & 1.0 & learning rate & 5.0e-4\\
		\hline
			learning rate schedule & constant & max steps & 6.0e6\\
		\hline
			memory size & 32 & normalize & true\\
		\hline
			num update & 1 & train interval & 5\\
		\hline
			num layers & 2 & time horizon & 64\\
		\hline
			sequence length & 128 & summary freq & 1000\\
		\hline
			tau & 0.005 & use recurrent & false \\
		\hline
			vis encode type & resnet & & \\
		\hline
			extrinsic strength & 1.0 & extrinsic gamma & 0.99\\
		\hline
			curiosity strength & 0.1 & curiosity gamma & 0.99\\
		\hline
			curiosity encoding size & 128 & & \\
		\hline
	\end{tabular}
\end{center}

Render Target Sensor: 32x32

Grayscale: false

Observation space size = 20

Action space size = 2

\vspace{2mm}

Episode steps = 1000

Total steps = 187000

Time = 6400s

\vspace{2mm}

Same training as before but with sensitivity of 10 on the Y-axis (has more range of accuracy). Already from the beginning it can be seen that the movements in Y are flatter than in X, I hope that it can adapt to larger ranges of movement in spite of the sensitivity. In next trainings we have to improve the reward system to penalize the distance in magnitude.

The reward curve is quite improved, even though we haven't included magnitude penalties yet.

In the editor, it needs the agent's observations to work well, although it does not become chaotic if left alone (sometimes it can end up looking at the zenith or the ground, but if the agent's observations are put back in, it recovers its position). It is also true that these cases have not yet been trained.

\section*{\color{green} 26/6 20:00 (6\textunderscore 26\textunderscore 2000)}

Trainer: SAC

\begin{center}
	\begin{tabular}{ | m{4cm} | m{2.5cm}||m{4cm} | m{2.5cm} | } 
		\hline
			batch size & 128 & buffer size & 200000 \\ 
		\hline
			buffer init steps & 5000 & hidden units & 256 \\
		\hline
			init entcoef & 1.0 & learning rate & 5.0e-4\\
		\hline
			learning rate schedule & constant & max steps & 6.0e6\\
		\hline
			memory size & 32 & normalize & true\\
		\hline
			num update & 1 & train interval & 5\\
		\hline
			num layers & 2 & time horizon & 64\\
		\hline
			sequence length & 128 & summary freq & 1000\\
		\hline
			tau & 0.005 & use recurrent & false \\
		\hline
			vis encode type & resnet & & \\
		\hline
			extrinsic strength & 1.0 & extrinsic gamma & 0.99\\
		\hline
			curiosity strength & 0.1 & curiosity gamma & 0.99\\
		\hline
			curiosity encoding size & 128 & & \\
		\hline
	\end{tabular}
\end{center}

Render Target Sensor: 32x32

Grayscale: false

Observation space size = 20

Action space size = 2

\vspace{2mm}

Episode steps = 1000

Total steps = 220000

Time = 8100s

\vspace{2mm}

A penalty has been added for failing the magnitude even if the angle is correct (it is 3 times less than punFactor, although it may have less influence on how the penalty is calculated).

The result is better than the previous ones, but taking into account that we still start from only bot observations. However, it doesn't seem to carry the Y-impulses very well (maybe because it doesn't see many, or because it has too much sensitivity).

Despite adding a penalty, it has achieved much higher rewards.

\section*{\color{green} 27/6 11:16 (6\textunderscore 27\textunderscore 1116)}

Trainer: BC

\begin{center}
	\begin{tabular}{ | m{4cm} | m{2.5cm}||m{4cm} | m{2.5cm} | } 
		\hline
			batches per epoch & 10 & batch size & 64 \\ 
		\hline
			hidden units & 128 & learning rate & 1.0e-5 \\ 
		\hline
			max steps & 5.0e4 & memory size & 32 \\ 
		\hline
			num layers & 4 & sequence length & 128 \\ 
		\hline
			summary freq & 1000 & use recurrent & false \\ 
		\hline
	\end{tabular}
\end{center}

\textbf{Demo path: demos/ComplexDemo.demo}

\vspace{2mm}

Render Target Sensor: 32x32

Grayscale: false

Observation space size = 20

Action space size = 2

\vspace{2mm}

Episode steps = 1000

Total steps = 37000

Time = 3400s

\vspace{2mm}

BC training and bot observations. It quickly adapts to the lines.

I should lower the sensitivity parameter in Y (10), and make new demos in which the position of the agent changes randomly (also modify Spawn Cases, so that it does not create planes while looking up/down). It doesn't take very well the impulses to the left at the end.

\section*{\color{red} 27/6 12:40 (6\textunderscore 27\textunderscore 1240)}

Trainer: BC

\begin{center}
	\begin{tabular}{ | m{4cm} | m{2.5cm}||m{4cm} | m{2.5cm} | } 
		\hline
			batches per epoch & 10 & batch size & 64 \\ 
		\hline
			hidden units & 128 & learning rate & 1.0e-5 \\ 
		\hline
			max steps & 5.0e4 & memory size & 32 \\ 
		\hline
			num layers & 4 & sequence length & 128 \\ 
		\hline
			summary freq & 1000 & use recurrent & false \\ 
		\hline
	\end{tabular}
\end{center}

Demo path: demos/ComplexDemo.demo

\vspace{2mm}

Render Target Sensor: 32x32

Grayscale: false

Observation space size = 20

Action space size = 2

\vspace{2mm}

Episode steps = 1000

Total steps = 9000

Time = 900s

\vspace{2mm}

Changing the observations from bot to agent at 6000 steps has stopped working well. Transitioning from one side to the other, it can be seen that the curve starts to flatten out.

\section*{\color{red} 27/6 12:57 (6\textunderscore 27\textunderscore 1257)}

Trainer: BC

\begin{center}
	\begin{tabular}{ | m{4cm} | m{2.5cm}||m{4cm} | m{2.5cm} | } 
		\hline
			batches per epoch & 10 & batch size & 64 \\ 
		\hline
			hidden units & 128 & learning rate & 1.0e-5 \\ 
		\hline
			max steps & 5.0e4 & memory size & 32 \\ 
		\hline
			num layers & 4 & sequence length & 128 \\ 
		\hline
			summary freq & 1000 & use recurrent & false \\ 
		\hline
	\end{tabular}
\end{center}

Demo path: demos/ComplexDemo.demo

\vspace{2mm}

Render Target Sensor: 32x32

Grayscale: false

Observation space size = 20

Action space size = 2

\vspace{2mm}

Episode steps = 1000

Total steps = 13000

Time = 1200s

\vspace{2mm}

Sensitivity in Y to 5 (also changed in the demo). Now start from the beginning with observations from the same agent. 

It has been adapting to a kind of midpoint (although it was doing some impulse) but at 9000 steps it already loses control.

\section*{\color{red} 27/6 13:31 (6\textunderscore 27\textunderscore 1331)}

Trainer: SAC

\begin{center}
	\begin{tabular}{ | m{4cm} | m{2.5cm}||m{4cm} | m{2.5cm} | } 
		\hline
			batch size & 128 & buffer size & 200000 \\ 
		\hline
			buffer init steps & 5000 & hidden units & 256 \\
		\hline
			init entcoef & 1.0 & learning rate & 5.0e-4\\
		\hline
			learning rate schedule & constant & max steps & 6.0e6\\
		\hline
			memory size & 32 & normalize & true\\
		\hline
			num update & 1 & train interval & 5\\
		\hline
			num layers & 2 & time horizon & 64\\
		\hline
			sequence length & 128 & summary freq & 1000\\
		\hline
			tau & 0.005 & use recurrent & false \\
		\hline
			vis encode type & resnet & & \\
		\hline
			extrinsic strength & 1.0 & extrinsic gamma & 0.99\\
		\hline
			curiosity strength & 0.1 & curiosity gamma & 0.99\\
		\hline
			curiosity encoding size & 128 & & \\
		\hline
			\textbf{gail strength} & \textbf{0.1} & \textbf{gail gamma} & \textbf{0.99}\\
		\hline
			\textbf{gail encoding size} & \textbf{128} & & \\
		\hline
	\end{tabular}
\end{center}

\textbf{Demo path: ./demos/ComplexDemo.demo}

\vspace{2mm}

Render Target Sensor: 32x32

Grayscale: false

Observation space size = 20

Action space size = 2

\vspace{2mm}

Episode steps = 1000

Total steps = 470000

Time = 18800s

\vspace{2mm}

Gail with agent's remarks. It takes it quite a few steps to get a good result, in the end it adapts to some of the curves but not with much precision. However, it usually gets the angle of the movements right (not so much the magnitude).

\section*{\color{red} 27/6 18:46 (6\textunderscore 27\textunderscore 1846)}

Trainer: SAC

\begin{center}
	\begin{tabular}{ | m{4cm} | m{2.5cm}||m{4cm} | m{2.5cm} | } 
		\hline
			batch size & 128 & buffer size & 200000 \\ 
		\hline
			buffer init steps & 5000 & hidden units & 256 \\
		\hline
			init entcoef & 1.0 & learning rate & 5.0e-4\\
		\hline
			learning rate schedule & constant & max steps & 6.0e6\\
		\hline
			memory size & 32 & normalize & true\\
		\hline
			num update & 1 & train interval & 5\\
		\hline
			num layers & 2 & time horizon & 64\\
		\hline
			sequence length & 128 & summary freq & 1000\\
		\hline
			tau & 0.005 & use recurrent & false \\
		\hline
			vis encode type & resnet & & \\
		\hline
			extrinsic strength & 1.0 & extrinsic gamma & 0.99\\
		\hline
			curiosity strength & 0.1 & curiosity gamma & 0.99\\
		\hline
			curiosity encoding size & 128 & & \\
		\hline
			\textbf{gail strength} & \textbf{0.05} & gail gamma & 0.99\\
		\hline
			gail encoding size & 128 & & \\
		\hline
	\end{tabular}
\end{center}

Demo path: ./demos/ComplexDemo.demo

\vspace{2mm}

Render Target Sensor: 32x32

Grayscale: false

Observation space size = 20

Action space size = 2

\vspace{2mm}

Episode steps = 1000

Total steps = 135000

Time = 5400s

\vspace{2mm}

Training with less GAIL, in this case with bot observations. It does not improve previous results, even with these observations.

\section*{\color{red} 27/6 20:54 (6\textunderscore 27\textunderscore 2054)}

Trainer: SAC

\begin{center}
	\begin{tabular}{ | m{4cm} | m{2.5cm}||m{4cm} | m{2.5cm} | } 
		\hline
			batch size & 128 & buffer size & 200000 \\ 
		\hline
			buffer init steps & 5000 & hidden units & 256 \\
		\hline
			init entcoef & 1.0 & learning rate & 5.0e-4\\
		\hline
			learning rate schedule & constant & max steps & 6.0e6\\
		\hline
			memory size & 32 & normalize & true\\
		\hline
			num update & 1 & train interval & 5\\
		\hline
			num layers & 2 & time horizon & 64\\
		\hline
			sequence length & 128 & summary freq & 1000\\
		\hline
			tau & 0.005 & use recurrent & false \\
		\hline
			vis encode type & resnet & & \\
		\hline
			extrinsic strength & 1.0 & extrinsic gamma & 0.99\\
		\hline
			curiosity strength & 0.1 & curiosity gamma & 0.99\\
		\hline
			curiosity encoding size & 128 & & \\
		\hline
	\end{tabular}
\end{center}

Render Target Sensor: 32x32

Grayscale: false

Observation space size = 20

Action space size = 2

\vspace{2mm}

\textbf{Episode steps = 500}

Total steps = 389000

Time = 13600s

\vspace{2mm}

In this training I will add the agent restart: at the end of each episode (which will be shorter than 1000).

I think I'll also have to restart the bot's movement with the agent reset to make it work.

Maybe the impulses created by restarting the agent are too big for the Y-sensitivity it has right now.

There are parts of the movement line to which it adapts very well, however it tends to ``anticipate'' the next impulse too soon. If I could get it to not finish them so soon it would be a very good result (even taking into account that we still have bot observations).

\section*{\color{red} 29/6 15:28 (6\textunderscore 29\textunderscore 1528)}

Trainer: PPO

\begin{center}
	\begin{tabular}{ | m{4cm} | m{2.5cm}||m{4cm} | m{2.5cm} | } 
		\hline
			batch size & 512 & beta & 5.0e-3 \\ 
		\hline
			buffer size & 2048 & epsilon & 0.2 \\
		\hline
		  hidden units & 256 & learning rate & 7.5e-4\\
		\hline
			learning rate schedule & linear & max steps & 6.0e5\\
		\hline
			memory size & 32 & normalize & false\\
		\hline
			num epoch & 4 & num layers & 2 \\
		\hline
			sequence length & 128 & summary freq & 1000 \\
		\hline
			time horizon & 32 & use recurrent & false\\
		\hline
			vis encode type & resnet & &\\
		\hline
			extrinsic strength & 1.0 & extrinsic gamma & 0.99\\
		\hline
			curiosity strength & 0.02 & curiosity gamma & 0.99\\
		\hline
			curiosity encoding size & 128 & & \\
		\hline
	\end{tabular}
\end{center}

Render Target Sensor: 32x32

Grayscale: false

Observation space size = 20

Action space size = 2

\vspace{2mm}

Episode steps = 500

Total steps = 337000

Time = 8000s

\vspace{2mm}

Training with cheat shaped tolerable range, the curves have experimental forms yet. RewFactor = PunFactor = 400. In this case it only creates targets on the right side (half in and half out).
I haven't changed the Y-sensitivity (still at 5) and that might affect the training somewhat.

It may be necessary to add a way that if it misses the shape at the end (anticipates too much the next move) it won't be rewarded later.
Also remove the random reset.

It doesn't mislead the angle of the move, but the reward stagnates  at 100 thousand steps.

\section*{\color{red} 29/6 17:48 (6\textunderscore 29\textunderscore 1748)}

Trainer: SAC

\begin{center}
	\begin{tabular}{ | m{4cm} | m{2.5cm}||m{4cm} | m{2.5cm} | } 
		\hline
			batch size & 128 & buffer size & 200000 \\ 
		\hline
			buffer init steps & 5000 & hidden units & 256 \\
		\hline
			init entcoef & 1.0 & learning rate & 5.0e-4\\
		\hline
			learning rate schedule & constant & max steps & 6.0e6\\
		\hline
			memory size & 32 & normalize & true\\
		\hline
			num update & 1 & train interval & 5\\
		\hline
			num layers & 2 & time horizon & 64\\
		\hline
			sequence length & 128 & summary freq & 1000\\
		\hline
			tau & 0.005 & use recurrent & false \\
		\hline
			vis encode type & resnet & & \\
		\hline
			extrinsic strength & 1.0 & extrinsic gamma & 0.99\\
		\hline
			curiosity strength & 0.1 & curiosity gamma & 0.99\\
		\hline
			curiosity encoding size & 128 & & \\
		\hline
	\end{tabular}
\end{center}

Render Target Sensor: 32x32

Grayscale: false

Observation space size = 20

Action space size = 2

\vspace{2mm}

\textbf{Episode steps = 1000}

Total steps = 100000

Time = 3600s

\vspace{2mm}

Random reboot and Y-sensitivity have been removed (although this may not have been necessary), and training is also done with SAC. The curves and factors are the same as before.

It tends to stay static in a straight line (as in some previous model), it hits the angle quite well but not the magnitude. I think it needs more penalty factor to be somewhat more effective.

\section*{\color{red} 29/6 18:53 (6\textunderscore 29\textunderscore 1853)}

Trainer: SAC

\begin{center}
	\begin{tabular}{ | m{4cm} | m{2.5cm}||m{4cm} | m{2.5cm} | } 
		\hline
			batch size & 128 & buffer size & 200000 \\ 
		\hline
			buffer init steps & 5000 & hidden units & 256 \\
		\hline
			init entcoef & 1.0 & learning rate & 5.0e-4\\
		\hline
			learning rate schedule & constant & max steps & 6.0e6\\
		\hline
			memory size & 32 & normalize & true\\
		\hline
			num update & 1 & train interval & 5\\
		\hline
			num layers & 2 & time horizon & 64\\
		\hline
			sequence length & 128 & summary freq & 1000\\
		\hline
			tau & 0.005 & use recurrent & false \\
		\hline
			vis encode type & resnet & & \\
		\hline
			extrinsic strength & 1.0 & extrinsic gamma & 0.99\\
		\hline
			curiosity strength & 0.1 & curiosity gamma & 0.99\\
		\hline
			curiosity encoding size & 128 & & \\
		\hline
	\end{tabular}
\end{center}

Render Target Sensor: 32x32

Grayscale: false

Observation space size = 20

Action space size = 2

\vspace{2mm}

Episode steps = 1000

Total steps = 124000

Time = 4500s

\vspace{2mm}

600 of punish factor and 250 of reward factor, the other parameters remain the same. It's not much better than before, probably because it's easier to get the initial impulse right than to try to gain something from the others. I'm going to try to make sure that if you don't get the last move right, you don't get rewarded on the next one.

\section*{\color{red} 29/6 20:18 (6\textunderscore 29\textunderscore 2018)}

Trainer: SAC

\begin{center}
	\begin{tabular}{ | m{4cm} | m{2.5cm}||m{4cm} | m{2.5cm} | } 
		\hline
			batch size & 128 & buffer size & 200000 \\ 
		\hline
			buffer init steps & 5000 & hidden units & 256 \\
		\hline
			init entcoef & 1.0 & learning rate & 5.0e-4\\
		\hline
			learning rate schedule & constant & max steps & 6.0e6\\
		\hline
			memory size & 32 & normalize & true\\
		\hline
			num update & 1 & train interval & 5\\
		\hline
			num layers & 2 & time horizon & 64\\
		\hline
			sequence length & 128 & summary freq & 1000\\
		\hline
			tau & 0.005 & use recurrent & false \\
		\hline
			vis encode type & resnet & & \\
		\hline
			extrinsic strength & 1.0 & extrinsic gamma & 0.99\\
		\hline
			curiosity strength & 0.1 & curiosity gamma & 0.99\\
		\hline
			curiosity encoding size & 128 & & \\
		\hline
	\end{tabular}
\end{center}

Render Target Sensor: 32x32

Grayscale: false

Observation space size = 20

Action space size = 2

\vspace{2mm}

Episode steps = 1000

Total steps = 498000

Time = 17600s

\vspace{2mm}

Corrections in the curves, it should work better now, although I haven't done the thing where I don't get rewarded on the first move if I miss the last one on the previous one.

Learn the angle correctly but not so much the magnitude, it might be necessary to add more exigency there (or a different curve). The amount of observations it receives could be reduced, since 10 previous movements are not necessary to have a clear idea of the movement (3-5 would be enough).

\section*{\color{red} 30/6 08:42 (6\textunderscore 30\textunderscore 0842)}

Trainer: SAC

\begin{center}
	\begin{tabular}{ | m{4cm} | m{2.5cm}||m{4cm} | m{2.5cm} | } 
		\hline
			batch size & 128 & buffer size & 200000 \\ 
		\hline
			buffer init steps & 5000 & hidden units & 256 \\
		\hline
			init entcoef & 1.0 & learning rate & 5.0e-4\\
		\hline
			learning rate schedule & constant & max steps & 6.0e6\\
		\hline
			memory size & 32 & normalize & true\\
		\hline
			num update & 1 & train interval & 5\\
		\hline
			num layers & 2 & time horizon & 64\\
		\hline
			sequence length & 128 & summary freq & 1000\\
		\hline
			tau & 0.005 & use recurrent & false \\
		\hline
			vis encode type & resnet & & \\
		\hline
			extrinsic strength & 1.0 & extrinsic gamma & 0.99\\
		\hline
			curiosity strength & 0.1 & curiosity gamma & 0.99\\
		\hline
			curiosity encoding size & 128 & & \\
		\hline
	\end{tabular}
\end{center}

Render Target Sensor: 32x32

Grayscale: false

\textbf{Observation space size = 6}

Action space size = 2

\vspace{2mm}

Episode steps = 1000

Total steps = 145000

Time = 5000s

\vspace{2mm}

Observations reduced to the last 3 of each axis. The way to detect targets has also been changed to isVisible. 

Momentum variable added, increases the reward if several are combined in a row and decreases it if not. PunFactor = 300 and RewFactor = 125, same curves.

It still matches the angle but not the magnitude.

\section*{\color{red} 30/6 10:13 (6\textunderscore 30\textunderscore 1013)}

Trainer: SAC

\begin{center}
	\begin{tabular}{ | m{4cm} | m{2.5cm}||m{4cm} | m{2.5cm} | } 
		\hline
			batch size & 128 & buffer size & 200000 \\ 
		\hline
			buffer init steps & 5000 & hidden units & 256 \\
		\hline
			init entcoef & 1.0 & learning rate & 5.0e-4\\
		\hline
			learning rate schedule & constant & max steps & 6.0e6\\
		\hline
			memory size & 32 & normalize & true\\
		\hline
			num update & 1 & train interval & 5\\
		\hline
			num layers & 2 & time horizon & 64\\
		\hline
			sequence length & 128 & summary freq & 1000\\
		\hline
			tau & 0.005 & use recurrent & false \\
		\hline
			vis encode type & resnet & & \\
		\hline
			extrinsic strength & 1.0 & extrinsic gamma & 0.99\\
		\hline
			curiosity strength & 0.1 & curiosity gamma & 0.99\\
		\hline
			curiosity encoding size & 128 & & \\
		\hline
	\end{tabular}
\end{center}

Render Target Sensor: 32x32

Grayscale: false

Observation space size = 6

Action space size = 2

\vspace{2mm}

Episode steps = 1000

Total steps = 71000

Time = 2400s

\vspace{2mm}

I removed a Mathf.Min that limited the penalties by magnitude. It should keep the spawner within the limits of the camera angle, because I'm seeing a lot of repeat shots.

\section*{\color{red} 30/6 10:43 (6\textunderscore 30\textunderscore 1043)}

Trainer: SAC

\begin{center}
	\begin{tabular}{ | m{4cm} | m{2.5cm}||m{4cm} | m{2.5cm} | } 
		\hline
			batch size & 128 & buffer size & 200000 \\ 
		\hline
			buffer init steps & 5000 & hidden units & 256 \\
		\hline
			init entcoef & 1.0 & learning rate & 5.0e-4\\
		\hline
			learning rate schedule & constant & max steps & 6.0e6\\
		\hline
			memory size & 32 & normalize & true\\
		\hline
			num update & 1 & train interval & 5\\
		\hline
			num layers & 2 & time horizon & 64\\
		\hline
			sequence length & 128 & summary freq & 1000\\
		\hline
			tau & 0.005 & use recurrent & false \\
		\hline
			vis encode type & resnet & & \\
		\hline
			extrinsic strength & 1.0 & extrinsic gamma & 0.99\\
		\hline
			curiosity strength & 0.1 & curiosity gamma & 0.99\\
		\hline
			curiosity encoding size & 128 & & \\
		\hline
	\end{tabular}
\end{center}

\textbf{Render Target Sensor: -}

\textbf{Grayscale: -}

Observation space size = 6

Action space size = 2

\vspace{2mm}

Episode steps = 1000

Total steps = 87000

Time = 1700s

\vspace{2mm}

Training without spawns and without a camera, the only objective is to imitate movement. All other parameters are the same as before. We will also have to review how the penalty is calculated with magnitude and think about increasing it. It follows the same curve, the next time I try to remove the factor that decreases the magnitude penalties.

\section*{\color{red} 30/6 11:32 (6\textunderscore 30\textunderscore 1132)}

Trainer: PPO

\begin{center}
	\begin{tabular}{ | m{4cm} | m{2.5cm}||m{4cm} | m{2.5cm} | } 
		\hline
			batch size & 512 & beta & 5.0e-3 \\ 
		\hline
			buffer size & 2048 & epsilon & 0.2 \\
		\hline
		  hidden units & 256 & learning rate & 7.5e-4\\
		\hline
			learning rate schedule & linear & max steps & 6.0e5\\
		\hline
			memory size & 32 & normalize & false\\
		\hline
			num epoch & 4 & num layers & 2 \\
		\hline
			sequence length & 128 & summary freq & 1000 \\
		\hline
			time horizon & 32 & use recurrent & false\\
		\hline
			vis encode type & resnet & &\\
		\hline
			extrinsic strength & 1.0 & extrinsic gamma & 0.99\\
		\hline
			curiosity strength & 0.02 & curiosity gamma & 0.99\\
		\hline
			curiosity encoding size & 128 & & \\
		\hline
	\end{tabular}
\end{center}

Render Target Sensor: -

Grayscale: -

Observation space size = 6

Action space size = 2

\vspace{2mm}

Episode steps = 1000

Total steps = 214000

Time = 4300s

\vspace{2mm}

Now there's no discount for failing the magnitude factor. I've also removed the momentum. It doesn't seem to be getting any better.

\section*{\color{red} 30/6 12:46 (6\textunderscore 30\textunderscore 1246)}

Trainer: SAC

\begin{center}
	\begin{tabular}{ | m{4cm} | m{2.5cm}||m{4cm} | m{2.5cm} | } 
		\hline
			batch size & 128 & buffer size & 200000 \\ 
		\hline
			buffer init steps & 5000 & hidden units & 256 \\
		\hline
			init entcoef & 1.0 & learning rate & 5.0e-4\\
		\hline
			learning rate schedule & constant & max steps & 6.0e6\\
		\hline
			memory size & 32 & normalize & true\\
		\hline
			num update & 1 & train interval & 5\\
		\hline
			num layers & 2 & time horizon & 64\\
		\hline
			sequence length & 128 & summary freq & 1000\\
		\hline
			tau & 0.005 & use recurrent & false \\
		\hline
			vis encode type & resnet & & \\
		\hline
			extrinsic strength & 1.0 & extrinsic gamma & 0.99\\
		\hline
			curiosity strength & 0.1 & curiosity gamma & 0.99\\
		\hline
			curiosity encoding size & 128 & & \\
		\hline
	\end{tabular}
\end{center}

Render Target Sensor: -

Grayscale: -

Observation space size = 6

Action space size = 2

\vspace{2mm}

Episode steps = 1000

Total steps = 123000

Time = 3000s

\vspace{2mm}

Same training as before but using SAC. If it doesn't improve we will have to rethink the shape of the curves or how to give more importance to the magnitude than to the angle. Same result.

\section*{\color{red} 30/6 13:37 (6\textunderscore 30\textunderscore 1337)}

Trainer: SAC

\begin{center}
	\begin{tabular}{ | m{4cm} | m{2.5cm}||m{4cm} | m{2.5cm} | } 
		\hline
			batch size & 128 & buffer size & 200000 \\ 
		\hline
			buffer init steps & 5000 & hidden units & 256 \\
		\hline
			init entcoef & 1.0 & learning rate & 5.0e-4\\
		\hline
			learning rate schedule & constant & max steps & 6.0e6\\
		\hline
			memory size & 32 & normalize & true\\
		\hline
			num update & 1 & train interval & 5\\
		\hline
			num layers & 2 & time horizon & 64\\
		\hline
			sequence length & 128 & summary freq & 1000\\
		\hline
			tau & 0.005 & use recurrent & false \\
		\hline
			vis encode type & resnet & & \\
		\hline
			extrinsic strength & 1.0 & extrinsic gamma & 0.99\\
		\hline
			curiosity strength & 0.1 & curiosity gamma & 0.99\\
		\hline
			curiosity encoding size & 128 & & \\
		\hline
	\end{tabular}
\end{center}

Render Target Sensor: -

Grayscale: -

Observation space size = 6

Action space size = 2

\vspace{2mm}

Episode steps = 1000

Total steps = 110000

Time = 2700s

\vspace{2mm}

More pronounced penalty curve. If it does not change, the tolerable range curve will probably have to be modified.

\section*{\color{red} 30/6 14:25 (6\textunderscore 30\textunderscore 1425)}

Trainer: SAC

\begin{center}
	\begin{tabular}{ | m{4cm} | m{2.5cm}||m{4cm} | m{2.5cm} | } 
		\hline
			batch size & 128 & buffer size & 200000 \\ 
		\hline
			buffer init steps & 5000 & hidden units & 256 \\
		\hline
			init entcoef & 1.0 & \textbf{learning rate} & \textbf{7.0e-4}\\
		\hline
			learning rate schedule & constant & max steps & 6.0e6\\
		\hline
			memory size & 32 & normalize & true\\
		\hline
			num update & 1 & train interval & 5\\
		\hline
			num layers & 2 & time horizon & 64\\
		\hline
			sequence length & 128 & summary freq & 1000\\
		\hline
			tau & 0.005 & use recurrent & false \\
		\hline
			vis encode type & resnet & & \\
		\hline
			extrinsic strength & 1.0 & extrinsic gamma & 0.99\\
		\hline
			curiosity strength & 0.1 & curiosity gamma & 0.99\\
		\hline
			curiosity encoding size & 128 & & \\
		\hline
	\end{tabular}
\end{center}

Render Target Sensor: -

Grayscale: -

Observation space size = 6

Action space size = 2

\vspace{2mm}

Episode steps = 1000

Total steps = 191000

Time = 4500s

\vspace{2mm}

Higher learning rate, 400 punFactor and 100 rewFactor.

\section*{\color{green} 30/6 15:42 (6\textunderscore 30\textunderscore 1542)}

Trainer: SAC

\begin{center}
	\begin{tabular}{ | m{4cm} | m{2.5cm}||m{4cm} | m{2.5cm} | } 
		\hline
			batch size & 128 & buffer size & 200000 \\ 
		\hline
			buffer init steps & 5000 & hidden units & 256 \\
		\hline
			init entcoef & 1.0 & learning rate & 7.0e-4\\
		\hline
			learning rate schedule & constant & max steps & 6.0e6\\
		\hline
			memory size & 32 & normalize & true\\
		\hline
			num update & 1 & train interval & 5\\
		\hline
			num layers & 2 & time horizon & 64\\
		\hline
			sequence length & 128 & summary freq & 1000\\
		\hline
			tau & 0.005 & use recurrent & false \\
		\hline
			vis encode type & resnet & & \\
		\hline
			extrinsic strength & 1.0 & extrinsic gamma & 0.99\\
		\hline
			curiosity strength & 0.1 & curiosity gamma & 0.99\\
		\hline
			curiosity encoding size & 128 & & \\
		\hline
	\end{tabular}
\end{center}

Render Target Sensor: -

Grayscale: -

Observation space size = 6

Action space size = 2

\vspace{2mm}

Episode steps = 1000

Total steps = 433000

Time = 11000s

\vspace{2mm}

Modified tolerable range curve, and some other modifications to the other 2. RewFactor=100, PunFactor=500.

Improved quite a bit from the previous ones.

\section*{\color{green} 30/6 18:53 (6\textunderscore 30\textunderscore 1853)}

Trainer: SAC

\begin{center}
	\begin{tabular}{ | m{4cm} | m{2.5cm}||m{4cm} | m{2.5cm} | } 
		\hline
			batch size & 128 & buffer size & 200000 \\ 
		\hline
			buffer init steps & 5000 & hidden units & 256 \\
		\hline
			init entcoef & 1.0 & learning rate & 7.0e-4\\
		\hline
			learning rate schedule & constant & max steps & 6.0e6\\
		\hline
			memory size & 32 & normalize & true\\
		\hline
			num update & 1 & train interval & 5\\
		\hline
			num layers & 2 & time horizon & 64\\
		\hline
			sequence length & 128 & summary freq & 1000\\
		\hline
			tau & 0.005 & use recurrent & false \\
		\hline
			vis encode type & resnet & & \\
		\hline
			extrinsic strength & 1.0 & extrinsic gamma & 0.99\\
		\hline
			curiosity strength & 0.1 & curiosity gamma & 0.99\\
		\hline
			curiosity encoding size & 128 & & \\
		\hline
	\end{tabular}
\end{center}

Render Target Sensor: -

Grayscale: -

Observation space size = 6

Action space size = 2

\vspace{2mm}

Episode steps = 1000

Total steps = 260000

Time = 6500s

\vspace{2mm}

Tolerable range curve flatter, small changes in the other 2. Rew=90, Pun=500.

\section*{\color{green} 30/6 20:44 (6\textunderscore 30\textunderscore 2044)}

Trainer: SAC

\begin{center}
	\begin{tabular}{ | m{4cm} | m{2.5cm}||m{4cm} | m{2.5cm} | } 
		\hline
			batch size & 128 & buffer size & 200000 \\ 
		\hline
			buffer init steps & 5000 & hidden units & 256 \\
		\hline
			init entcoef & 1.0 & learning rate & 7.0e-4\\
		\hline
			learning rate schedule & constant & max steps & 6.0e6\\
		\hline
			memory size & 32 & normalize & true\\
		\hline
			num update & 1 & train interval & 5\\
		\hline
			num layers & 2 & time horizon & 64\\
		\hline
			sequence length & 128 & summary freq & 1000\\
		\hline
			tau & 0.005 & use recurrent & false \\
		\hline
			vis encode type & resnet & & \\
		\hline
			extrinsic strength & 1.0 & extrinsic gamma & 0.99\\
		\hline
			curiosity strength & 0.1 & curiosity gamma & 0.99\\
		\hline
			curiosity encoding size & 128 & & \\
		\hline
	\end{tabular}
\end{center}

Render Target Sensor: -

Grayscale: -

Observation space size = 6

Action space size = 2

\vspace{2mm}

Episode steps = 1000

Total steps = 339000

Time = 8000s

\vspace{2mm}

Good result like the previous ones, has to anticipate the next movements yet.

\section*{\color{green} 30/6 23:00 (6\textunderscore 30\textunderscore 2300)}

Trainer: SAC

\begin{center}
	\begin{tabular}{ | m{4cm} | m{2.5cm}||m{4cm} | m{2.5cm} | } 
		\hline
			batch size & 128 & buffer size & 200000 \\ 
		\hline
			buffer init steps & 5000 & hidden units & 256 \\
		\hline
			init entcoef & 1.0 & learning rate & 7.0e-4\\
		\hline
			learning rate schedule & constant & max steps & 6.0e6\\
		\hline
			memory size & 32 & normalize & true\\
		\hline
			num update & 1 & train interval & 5\\
		\hline
			num layers & 2 & time horizon & 64\\
		\hline
			sequence length & 128 & summary freq & 1000\\
		\hline
			tau & 0.005 & use recurrent & false \\
		\hline
			vis encode type & resnet & & \\
		\hline
			extrinsic strength & 1.0 & extrinsic gamma & 0.99\\
		\hline
			curiosity strength & 0.1 & curiosity gamma & 0.99\\
		\hline
			curiosity encoding size & 128 & & \\
		\hline
	\end{tabular}
\end{center}

Render Target Sensor: -

Grayscale: -

Observation space size = 6

Action space size = 2

\vspace{2mm}

Episode steps = 1000

Total steps = 223000

Time = 5600s

\vspace{2mm}

More reward at the end of the moves, the rest of the curves and parameters remain the same. It manages to make some pretty good moves, but it's far from perfect. In next practice I could try to reverse the reward curve to see how it acts, and use PPO with these rewards.

\section*{\color{red} 01/7 09:03 (7\textunderscore 01\textunderscore 0903)}

Trainer: PPO

\begin{center}
	\begin{tabular}{ | m{4cm} | m{2.5cm}||m{4cm} | m{2.5cm} | } 
		\hline
			batch size & 512 & beta & 5.0e-3 \\ 
		\hline
			buffer size & 2048 & epsilon & 0.2 \\
		\hline
		  hidden units & 256 & learning rate & 7.5e-4\\
		\hline
			learning rate schedule & linear & max steps & 6.0e5\\
		\hline
			memory size & 32 & normalize & false\\
		\hline
			num epoch & 4 & num layers & 2 \\
		\hline
			sequence length & 128 & summary freq & 1000 \\
		\hline
			time horizon & 32 & use recurrent & false\\
		\hline
			vis encode type & resnet & &\\
		\hline
			extrinsic strength & 1.0 & extrinsic gamma & 0.99\\
		\hline
			curiosity strength & 0.02 & curiosity gamma & 0.99\\
		\hline
			curiosity encoding size & 128 & & \\
		\hline
	\end{tabular}
\end{center}

Render Target Sensor: -

Grayscale: -

Observation space size = 6

Action space size = 2

\vspace{2mm}

Episode steps = 1000

Total steps = 309000

Time = 6300s

\vspace{2mm}

Same training as 6\textunderscore 30\textunderscore 2300, but using PPO instead of SAC. Compare both models at the end.

The result is much worse than with SAC. It is saved for comparison.

\section*{\color{red} 01/7 10:53 (7\textunderscore 01\textunderscore 1053)}

Trainer: SAC

\begin{center}
	\begin{tabular}{ | m{4cm} | m{2.5cm}||m{4cm} | m{2.5cm} | } 
		\hline
			batch size & 128 & buffer size & 200000 \\ 
		\hline
			buffer init steps & 5000 & hidden units & 256 \\
		\hline
			init entcoef & 1.0 & learning rate & 7.0e-4\\
		\hline
			learning rate schedule & constant & max steps & 6.0e6\\
		\hline
			memory size & 32 & normalize & true\\
		\hline
			num update & 1 & train interval & 5\\
		\hline
			num layers & 2 & time horizon & 64\\
		\hline
			sequence length & 128 & summary freq & 1000\\
		\hline
			tau & 0.005 & use recurrent & false \\
		\hline
			vis encode type & resnet & & \\
		\hline
			extrinsic strength & 1.0 & extrinsic gamma & 0.99\\
		\hline
			curiosity strength & 0.1 & curiosity gamma & 0.99\\
		\hline
			curiosity encoding size & 128 & & \\
		\hline
	\end{tabular}
\end{center}

Render Target Sensor: -

Grayscale: -

Observation space size = 6

Action space size = 2

\vspace{2mm}

Episode steps = 1000

Total steps = 78000

Time = 1900s

\vspace{2mm}

Reward factor curve with higher value at the end (in addition to the beginning). The factors remain at: rew=90, pun=500.

It tends to stay flat at the top (although it does some concrete moves well).

\section*{\color{red} 01/7 11:33 (7\textunderscore 01\textunderscore 1133)}

Trainer: SAC

\begin{center}
	\begin{tabular}{ | m{4cm} | m{2.5cm}||m{4cm} | m{2.5cm} | } 
		\hline
			batch size & 128 & buffer size & 200000 \\ 
		\hline
			buffer init steps & 5000 & hidden units & 256 \\
		\hline
			init entcoef & 1.0 & learning rate & 7.0e-4\\
		\hline
			learning rate schedule & constant & max steps & 6.0e6\\
		\hline
			memory size & 32 & normalize & true\\
		\hline
			num update & 1 & train interval & 5\\
		\hline
			num layers & 2 & time horizon & 64\\
		\hline
			sequence length & 128 & summary freq & 1000\\
		\hline
			tau & 0.005 & use recurrent & false \\
		\hline
			vis encode type & resnet & & \\
		\hline
			extrinsic strength & 1.0 & extrinsic gamma & 0.99\\
		\hline
			curiosity strength & 0.1 & curiosity gamma & 0.99\\
		\hline
			curiosity encoding size & 128 & & \\
		\hline
	\end{tabular}
\end{center}

Render Target Sensor: -

Grayscale: -

Observation space size = 6

Action space size = 2

\vspace{2mm}

Episode steps = 1000

Total steps = 63000

Time = 1500s

\vspace{2mm}

Flatter tolerable range curve, earlier shape rewards curve and slightly longer penalty curve. Still using momentum, now I've put a variable to control the relationship between the magnitude penalties (the higher it is, the more its importance is reduced, and if it's lower than 1 it's given more value than the angle, for now it's 1).

It seems to be worse than the cases with similar curves.

\section*{\color{green} 01/7 12:01 (7\textunderscore 01\textunderscore 1201)}

Trainer: SAC

\begin{center}
	\begin{tabular}{ | m{4cm} | m{2.5cm}||m{4cm} | m{2.5cm} | } 
		\hline
			batch size & 128 & buffer size & 200000 \\ 
		\hline
			buffer init steps & 5000 & hidden units & 256 \\
		\hline
			init entcoef & 1.0 & learning rate & 7.0e-4\\
		\hline
			learning rate schedule & constant & max steps & 6.0e6\\
		\hline
			memory size & 32 & normalize & true\\
		\hline
			num update & 1 & train interval & 5\\
		\hline
			num layers & 2 & time horizon & 64\\
		\hline
			sequence length & 128 & summary freq & 1000\\
		\hline
			tau & 0.005 & use recurrent & false \\
		\hline
			vis encode type & resnet & & \\
		\hline
			extrinsic strength & 1.0 & extrinsic gamma & 0.99\\
		\hline
			curiosity strength & 0.1 & curiosity gamma & 0.99\\
		\hline
			curiosity encoding size & 128 & & \\
		\hline
	\end{tabular}
\end{center}

Render Target Sensor: -

Grayscale: -

Observation space size = 6

Action space size = 2

\vspace{2mm}

Episode steps = 1000

Total steps = 221000

Time = 5500s

\vspace{2mm}

Tolerable range curve scaled up (as before), and the penalty curve restored to the previous shape. The magnitude factor is now 0.75, so it penalizes more.

Pretty good result.

In the next practices I will use 3 more curves to save the best ones.

\section*{\color{green} 01/7 13:41 (7\textunderscore 01\textunderscore 1341)}

Trainer: SAC

\begin{center}
	\begin{tabular}{ | m{4cm} | m{2.5cm}||m{4cm} | m{2.5cm} | } 
		\hline
			batch size & 128 & buffer size & 200000 \\ 
		\hline
			buffer init steps & 5000 & hidden units & 256 \\
		\hline
			init entcoef & 1.0 & learning rate & 7.0e-4\\
		\hline
			learning rate schedule & constant & max steps & 6.0e6\\
		\hline
			memory size & 32 & normalize & true\\
		\hline
			num update & 1 & train interval & 5\\
		\hline
			num layers & 2 & time horizon & 64\\
		\hline
			sequence length & 128 & summary freq & 1000\\
		\hline
			tau & 0.005 & use recurrent & false \\
		\hline
			vis encode type & resnet & & \\
		\hline
			extrinsic strength & 1.0 & extrinsic gamma & 0.99\\
		\hline
			curiosity strength & 0.1 & curiosity gamma & 0.99\\
		\hline
			curiosity encoding size & 128 & & \\
		\hline
	\end{tabular}
\end{center}

Render Target Sensor: -

Grayscale: -

Observation space size = 6

Action space size = 2

\vspace{2mm}

Episode steps = 1000

Total steps = 245000

Time = 6200s

\vspace{2mm}

Tolerable range curve with more value at the end, sharper reward curve (only the first move with more value counts practically) and less lengthy penalty curve. Pun=500 and Rew=75.

The result has improved, the next one I'll add more pressure to the penalties.

\section*{\color{red} 01/7 15:29 (7\textunderscore 01\textunderscore 1529)}

Trainer: SAC

\begin{center}
	\begin{tabular}{ | m{4cm} | m{2.5cm}||m{4cm} | m{2.5cm} | } 
		\hline
			batch size & 128 & buffer size & 200000 \\ 
		\hline
			buffer init steps & 5000 & hidden units & 256 \\
		\hline
			init entcoef & 1.0 & learning rate & 7.0e-4\\
		\hline
			learning rate schedule & constant & max steps & 6.0e6\\
		\hline
			memory size & 32 & normalize & true\\
		\hline
			num update & 1 & train interval & 5\\
		\hline
			num layers & 2 & time horizon & 64\\
		\hline
			sequence length & 128 & summary freq & 1000\\
		\hline
			tau & 0.005 & use recurrent & false \\
		\hline
			vis encode type & resnet & & \\
		\hline
			extrinsic strength & 1.0 & extrinsic gamma & 0.99\\
		\hline
			curiosity strength & 0.1 & curiosity gamma & 0.99\\
		\hline
			curiosity encoding size & 128 & & \\
		\hline
	\end{tabular}
\end{center}

Render Target Sensor: -

Grayscale: -

Observation space size = 6

Action space size = 2

\vspace{2mm}

Episode steps = 1000

Total steps = 118000

Time = 3000s

\vspace{2mm}

Rew=70, Pun=600. Same curves as in previous practice.

Looks slightly worse than the last one.

\section*{\color{green} 01/7 16:23 (7\textunderscore 01\textunderscore 1623)}

Trainer: SAC

\begin{center}
	\begin{tabular}{ | m{4cm} | m{2.5cm}||m{4cm} | m{2.5cm} | } 
		\hline
			batch size & 128 & buffer size & 200000 \\ 
		\hline
			buffer init steps & 5000 & hidden units & 256 \\
		\hline
			init entcoef & 1.0 & learning rate & 7.0e-4\\
		\hline
			learning rate schedule & constant & max steps & 6.0e6\\
		\hline
			memory size & 32 & normalize & true\\
		\hline
			num update & 1 & train interval & 5\\
		\hline
			num layers & 2 & time horizon & 64\\
		\hline
			sequence length & 128 & summary freq & 1000\\
		\hline
			tau & 0.005 & use recurrent & false \\
		\hline
			vis encode type & resnet & & \\
		\hline
			extrinsic strength & 1.0 & extrinsic gamma & 0.99\\
		\hline
			curiosity strength & 0.1 & curiosity gamma & 0.99\\
		\hline
			curiosity encoding size & 128 & & \\
		\hline
	\end{tabular}
\end{center}

Render Target Sensor: -

Grayscale: -

Observation space size = 6

Action space size = 2

\vspace{2mm}

Episode steps = 1000

Total steps = 195000

Time = 5000s

\vspace{2mm}

Tolerable reduced range, but also the penalty factor: PunFactor=400, RewFactor=100. 

Correct result, similar to the previous ones (not saved).

\section*{\color{red} 01/7 17:48 (7\textunderscore 01\textunderscore 1748)}

Trainer: SAC

\begin{center}
	\begin{tabular}{ | m{4cm} | m{2.5cm}||m{4cm} | m{2.5cm} | } 
		\hline
			batch size & 128 & buffer size & 200000 \\ 
		\hline
			buffer init steps & 5000 & hidden units & 256 \\
		\hline
			init entcoef & 1.0 & learning rate & 7.0e-4\\
		\hline
			learning rate schedule & constant & max steps & 6.0e6\\
		\hline
			memory size & 32 & normalize & true\\
		\hline
			num update & 1 & train interval & 5\\
		\hline
			num layers & 2 & time horizon & 64\\
		\hline
			sequence length & 128 & summary freq & 1000\\
		\hline
			tau & 0.005 & use recurrent & false \\
		\hline
			vis encode type & resnet & & \\
		\hline
			extrinsic strength & 1.0 & extrinsic gamma & 0.99\\
		\hline
			curiosity strength & 0.1 & curiosity gamma & 0.99\\
		\hline
			curiosity encoding size & 128 & & \\
		\hline
	\end{tabular}
\end{center}

Render Target Sensor: -

Grayscale: -

Observation space size = 6

Action space size = 2

\vspace{2mm}

Episode steps = 1000

Total steps = 95000

Time = 2400s

\vspace{2mm}

Training with the tolerable range demanding of the last one, but factors of 500-75. In addition, the magnitude ratio is 0.5 so it penalizes even more. It does not improve the previous results, the best model so far seems to be 7\textunderscore 01\textunderscore 1341.

\section*{\color{red} 01/7 18:51 (7\textunderscore 01\textunderscore 1851)}

Trainer: SAC

\begin{center}
	\begin{tabular}{ | m{4cm} | m{2.5cm}||m{4cm} | m{2.5cm} | } 
		\hline
			batch size & 128 & buffer size & 200000 \\ 
		\hline
			buffer init steps & 5000 & hidden units & 256 \\
		\hline
			init entcoef & 1.0 & learning rate & 7.0e-4\\
		\hline
			learning rate schedule & constant & max steps & 6.0e6\\
		\hline
			memory size & 32 & normalize & true\\
		\hline
			num update & 1 & train interval & 5\\
		\hline
			num layers & 2 & time horizon & 64\\
		\hline
			sequence length & 128 & summary freq & 1000\\
		\hline
			tau & 0.005 & use recurrent & false \\
		\hline
			vis encode type & resnet & & \\
		\hline
			extrinsic strength & 1.0 & extrinsic gamma & 0.99\\
		\hline
			curiosity strength & 0.1 & curiosity gamma & 0.99\\
		\hline
			curiosity encoding size & 128 & & \\
		\hline
	\end{tabular}
\end{center}

Render Target Sensor: -

Grayscale: -

Observation space size = 6

Action space size = 2

\vspace{2mm}

Episode steps = 1000

Total steps = 73000

Time = 1800s

\vspace{2mm}

Same parameters as in 7\textunderscore 01\textunderscore 1341 (Pun=500, Rew=75, useMomentum, magnitude=0.75) but using as space actions the angle of the movement and its magnitude (normalized). I could create a vector of observations that had the last angles and magnitudes (instead of x and y).

The result seems to improve at times, in future trainings use angle and magnitude observations.

\section*{\color{red} 01/7 19:31 (7\textunderscore 01\textunderscore 1931)}

Trainer: SAC

\begin{center}
	\begin{tabular}{ | m{4cm} | m{2.5cm}||m{4cm} | m{2.5cm} | } 
		\hline
			batch size & 128 & buffer size & 200000 \\ 
		\hline
			buffer init steps & 5000 & hidden units & 256 \\
		\hline
			init entcoef & 1.0 & learning rate & 7.0e-4\\
		\hline
			learning rate schedule & constant & max steps & 6.0e6\\
		\hline
			memory size & 32 & normalize & true\\
		\hline
			num update & 1 & train interval & 5\\
		\hline
			num layers & 2 & time horizon & 64\\
		\hline
			sequence length & 128 & summary freq & 1000\\
		\hline
			tau & 0.005 & use recurrent & false \\
		\hline
			vis encode type & resnet & & \\
		\hline
			extrinsic strength & 1.0 & extrinsic gamma & 0.99\\
		\hline
			curiosity strength & 0.1 & curiosity gamma & 0.99\\
		\hline
			curiosity encoding size & 128 & & \\
		\hline
	\end{tabular}
\end{center}

Render Target Sensor: -

Grayscale: -

Observation space size = 6

Action space size = 2

\vspace{2mm}

Episode steps = 1000

Total steps = 143000

Time = 3400s

\vspace{2mm}

Now the observations are \textbf{angle-magnitude} (which are also size 6). In theory the problem should be rather more trivial for the neural network, even if the numbers saved from observation are not the exact actions (they are saved already processed and normalized).

It makes some good angles, but only underneath. This may be because at angle 0 the action angles -1 and +1 coincide.

\section*{\color{red} 01/7 20:44 (7\textunderscore 01\textunderscore 2044)}

Trainer: SAC

\begin{center}
	\begin{tabular}{ | m{4cm} | m{2.5cm}||m{4cm} | m{2.5cm} | } 
		\hline
			batch size & 128 & buffer size & 200000 \\ 
		\hline
			buffer init steps & 5000 & hidden units & 256 \\
		\hline
			init entcoef & 1.0 & learning rate & 7.0e-4\\
		\hline
			learning rate schedule & constant & max steps & 6.0e6\\
		\hline
			memory size & 32 & normalize & true\\
		\hline
			num update & 1 & train interval & 5\\
		\hline
			num layers & 2 & time horizon & 64\\
		\hline
			sequence length & 128 & summary freq & 1000\\
		\hline
			tau & 0.005 & use recurrent & false \\
		\hline
			vis encode type & resnet & & \\
		\hline
			extrinsic strength & 1.0 & extrinsic gamma & 0.99\\
		\hline
			curiosity strength & 0.1 & curiosity gamma & 0.99\\
		\hline
			curiosity encoding size & 128 & & \\
		\hline
	\end{tabular}
\end{center}

Render Target Sensor: -

Grayscale: -

Observation space size = 6

Action space size = 2

\vspace{2mm}

Episode steps = 1000

Total steps = 62000

Time = 1500s

\vspace{2mm}

Flat curves. The direction of movement in this bot has been reversed (now it goes to the left). The observations have been modified to save the action before processing (keep in mind when testing the agent 7\textunderscore 01\textunderscore 1931).

The movement remains flat, these curves are not useful.

\section*{\color{red} 01/7 21:14 (7\textunderscore 01\textunderscore 2114)}

Trainer: SAC

\begin{center}
	\begin{tabular}{ | m{4cm} | m{2.5cm}||m{4cm} | m{2.5cm} | } 
		\hline
			batch size & 128 & buffer size & 200000 \\ 
		\hline
			buffer init steps & 5000 & hidden units & 256 \\
		\hline
			init entcoef & 1.0 & learning rate & 7.0e-4\\
		\hline
			learning rate schedule & constant & max steps & 6.0e6\\
		\hline
			memory size & 32 & normalize & true\\
		\hline
			num update & 1 & train interval & 5\\
		\hline
			num layers & 2 & time horizon & 64\\
		\hline
			sequence length & 128 & summary freq & 1000\\
		\hline
			tau & 0.005 & use recurrent & false \\
		\hline
			vis encode type & resnet & & \\
		\hline
			extrinsic strength & 1.0 & extrinsic gamma & 0.99\\
		\hline
			curiosity strength & 0.1 & curiosity gamma & 0.99\\
		\hline
			curiosity encoding size & 128 & & \\
		\hline
	\end{tabular}
\end{center}

Render Target Sensor: -

Grayscale: -

Observation space size = 6

Action space size = 2

\vspace{2mm}

Episode steps = 1000

Total steps = 94000

Time = 2400s

\vspace{2mm}

Modified curves similar to the first ones, pun=600 and rew=75.

The curve stays flat on the average, it doesn't work.
