% ==============================================================================
% This file is part of the "LaTeX template for writing the Final Degree Work
% report". It has been developed to aid the students of the Bachelor's Degree in
% Video Game Design and Development at the Jaume I University.
%
% (c) 2019 Sergio Barrachina Mir and José Vte. Martí Avilés
%
% The template can be used and distributed under the next license:
%  Creative Commons Attribution-NonCommercial-ShareAlike (CC BY-NC-SA)
%  http://creativecommons.org/licenses/by-nc-sa/3.0/
%  http://creativecommons.org/licenses/by-nc-sa/3.0/legalcode
%
% Atom editor configuration follows:
% !TEX root = ./report.tex
% !TeX spellcheck = en-US
% ==============================================================================


%======================================================================
% Packages
%----------------------------------------------------------------------
\usepackage[scale=.7]{geometry}
\usepackage[utf8]{inputenc}
\usepackage[spanish,english]{babel}
\usepackage[pdftex]{graphicx}
\usepackage{booktabs}
\usepackage[usenames,dvipsnames]{color}
\usepackage{listings}
\usepackage{colortbl}
\usepackage{pdfpages}          % required for chapter
\usepackage{soul}              % required for chapter
\usepackage[pdftex]{hyperref}  % hyperref must be loaded before minitoc
\usepackage{titling}
%======================================================================


%======================================================================
% Font and font encoding
%----------------------------------------------------------------------
\usepackage[T1]{fontenc}
\usepackage{lmodern}               % replacement for computer modern
\usepackage[scaled=.8]{beramono}   % beramono monospaced font
\usepackage{scalefnt}              % Allows font escalation
% Next two lines avoid the "Font shape `T1/lmr/bx/sc' undefined" error
\normalfont %to load T1lmr.fd
\DeclareFontShape{T1}{lmr}{bx}{sc} { <-> ssub * cmr/bx/sc }{}
%======================================================================


%======================================================================
% hyperref configuration
%----------------------------------------------------------------------
\definecolor{darkblue}{rgb}{.0,.0,.3}
\hypersetup{
 colorlinks=true,
 linkcolor=black,   % Color for normal internal links.
 anchorcolor=black, % Color for anchor text.
 citecolor=black,   % Color for bibligraphical citations in text.
 filecolor=black,   % Color for URLs which open local files.
 menucolor=black,   % Color for Acrobat menu items.
 urlcolor=darkblue,     % Color for linked URLs.
}
%======================================================================


%======================================================================
% tableofcontents configuration
%----------------------------------------------------------------------
\setsecnumdepth{subsection}
%======================================================================


%======================================================================
% Chapter style
%----------------------------------------------------------------------
\definecolor{nicered}{rgb}{.129,.129,.129}
\makeatletter
\newlength\dlf@normtxtw
\setlength\dlf@normtxtw{\textwidth}
\def\myhelvetfont{\def\sfdefault{mdput}}
\newsavebox{\feline@chapter}
\newcommand\feline@chapter@marker[1][4cm]{%
 \sbox\feline@chapter{%
  \resizebox{!}{#1}{\fboxsep=1pt%
   \colorbox{nicered}{\color{white}\bfseries\sffamily\thechapter}%
  }}%
 \rotatebox{90}{%
  \resizebox{%
   \heightof{\usebox{\feline@chapter}}+\depthof{\usebox{\feline@chapter}}}%
  {!}{\scshape\so\@chapapp}}\quad%
 \raisebox{\depthof{\usebox{\feline@chapter}}}{\usebox{\feline@chapter}}%
}
\newcommand\feline@chm[1][4cm]{%
 \sbox\feline@chapter{\feline@chapter@marker[#1]}%
 \makebox[0pt][l]{% aka \rlap
  \makebox[0cm][r]{\usebox\feline@chapter}%
 }}
\makechapterstyle{daleif1}{
 \renewcommand\chapnamefont{\normalfont\Large\scshape\raggedleft\so}
 \renewcommand\chaptitlefont{\normalfont\huge\bfseries\scshape\color{nicered}}
 \renewcommand\chapternamenum{}
 \renewcommand\printchaptername{}
 \renewcommand\printchapternum{\null\hfill\feline@chm[2.5cm]\par}
 \renewcommand\afterchapternum{\par\vskip\midchapskip}
 \renewcommand\printchaptertitle[1]{\chaptitlefont\raggedleft ##1\par}
}
\makeatother
\chapterstyle{daleif1}
%======================================================================


%======================================================================
% Headers and footers
%----------------------------------------------------------------------
% \makepsmarks{mycompanion}{%
%   \let\@mkboth\markboth
%   \def\chaptermark##1{\markboth{##1}{##1}}% % left & right marks
%   \def\sectionmark##1{\markright{% % right mark
%       \ifnum \c@secnumdepth>\z@
%       \thesection. \% % section number
%       \fi
%       ##1}}
% }
\makeevenhead{companion}%
{\small\thepage}{}{\small\leftmark}
\makeoddhead{companion}%
{\small\rightmark}{}{\small\thepage}
\pagestyle{companion}
%======================================================================


%======================================================================
% minitoc package and configuration
%----------------------------------------------------------------------
\usepackage[tight,checkfiles]{minitoc}
\setcounter{minitocdepth}{1}
\mtcselectlanguage{english}
%\renewcommand{\mtctitle}{Índice}
%======================================================================


%======================================================================
% maketitle customization
%----------------------------------------------------------------------
\newlength{\myadjustlength}
\pretitle{%
  \begin{center}

    \huge

    \vspace{-3cm}

    \includegraphics[width=3.5cm]{img/logo-universitat-jaume-i}

    \bigskip{}

    \bigskip{}

    \bfseries
}

\posttitle{
  \par\end{center}
  \medskip{}
}

\preauthor{
  \begin{center}
    \large
    \bfseries
    \begin{tabular}[t]{c}
}

\postauthor{
    \end{tabular}
    \normalfont

    \bigskip{}
    \course{}

    \smallskip{}
    \degree{}

    \smallskip{}
    Universitat Jaume I

  \end{center}
}

\predate{
  \begin{center}
    \large
}

\postdate{
    \par\end{center}

    \vfill

    \hfill{}Supervised by: \director{}

  \begin{picture}(0,0)
    \put(400,-110){\href{http://creativecommons.org/licenses/by-nc-sa/3.0/}{\includegraphics[scale=.8]{img/by-nc-sa-eu.pdf}}}
  \end{picture}

  \thispagestyle{empty}

  \newpage{}

  \null{}

  \thispagestyle{empty}
}


\makeatletter
\newcommand{\maketitlebbb}{%
 \thispagestyle{empty}

 \calccentering{\myadjustlength}
 \begin{adjustwidth*}{\myadjustlength}{-\myadjustlength}

  \null


  \begin{center}

   \includegraphics{img/logo-universitat-jaume-i}

   \bigskip{}

   {\sffamily \bfseries \huge \@title \par}

   \bigskip

   {\sffamily \Large \@author}

   \bigskip

   {\sffamily \large
    \course

    \smallskip

    \degree

    \smallskip

    Universitat Jaume I
   }

   \bigskip

   {\sffamily \Large \@date}


  \end{center}

  \vfill

  \vfill

  \vfill

  \hfill{}{\sffamily \large Work supervised by: \director}

 \end{adjustwidth*}

 \begin{picture}(0,0)
  \put(400,-110){\href{http://creativecommons.org/licenses/by-nc-sa/3.0/}{\includegraphics[scale=.8]{img/by-nc-sa-eu.pdf}}}
 \end{picture}

 \newpage{}

 \null{}

 \thispagestyle{empty}
}
\makeatother
%======================================================================


%======================================================================
% Colors for the source code stuff
%----------------------------------------------------------------------
\definecolor{codecolor}{rgb}{0,0,0.3}
\definecolor{gray97}{gray}{.97}
\definecolor{gray45}{gray}{.45}
%======================================================================


%======================================================================
% Source code stuff
%----------------------------------------------------------------------
% Which languages to load
\lstloadlanguages{C, C++, Java, Python}
% Listings settings
\lstset{ %
frame=Ltb,
framerule=0pt,
framextopmargin=3pt,
framexbottommargin=3pt,
framexleftmargin=3pt,
framesep=0pt,
rulesep=.4pt,
backgroundcolor=\color{gray97},
rulesepcolor=\color{black},
%
stringstyle=\ttfamily,
showstringspaces = true,
%
basicstyle=\ttfamily,
commentstyle=\color{gray45},
keywordstyle=\color{codecolor}\bfseries,
%
numbers=left,
numbersep=7pt,
numberstyle=\tiny,
numberfirstline = false,
%
belowcaptionskip = 0pt,
%
literate=
 {á}{{\'a}}1
{é}{{\'e}}1
{í}{{\'i}}1
{ó}{{\'o}}1
{ú}{{\'u}}1
{ñ}{{\~n}}1
{Á}{{\'A}}1
{É}{{\'E}}1
{Í}{{\'I}}1
{Ó}{{\'O}}1
{Ú}{{\'U}}1
{Ñ}{{\~N}}1
{¡}{{!`}}1
{¿}{{?`}}1
{«}{{\guillemotleft{}}}1
{»}{{\guillemotright{}}}1
}
%======================================================================


%======================================================================
% For long urls in bibliography
%----------------------------------------------------------------------
\Urlmuskip = 0mu plus 1mu
%======================================================================
