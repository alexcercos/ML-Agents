% ==============================================================================
% This file is part of the "LaTeX template for writing the Final Degree Work
% report". It has been developed to aid the students of the Bachelor's Degree in
% Video Game Design and Development at the Jaume I University.
%
% (c) 2019 Sergio Barrachina Mir and José Vte. Martí Avilés
%
% The template can be used and distributed under the next license:
%  Creative Commons Attribution-NonCommercial-ShareAlike (CC BY-NC-SA)
%  http://creativecommons.org/licenses/by-nc-sa/3.0/
%  http://creativecommons.org/licenses/by-nc-sa/3.0/legalcode
%
% Atom editor configuration follows:
% !TEX root = ./report.tex
% !TeX spellcheck = en-US
% ==============================================================================

\chapter{Introduction}

\minitoc{}

\bigskip{}

In this document, we will detail the steps for the realization of a neural network model capable of imitating real player behaviors in simple games, from the programming of the game that we will use as a test case to the analysis of results.

To obtain the dataset, we will use Unity3D to program a shooter-type game with no player movement on the stage (Point-and-Click), and random targets. In order to obtain a reasonable dataset to train the neural network, an NPC behaviour will be programmed to simulate a large amount of games as the player. That NPC would have recognizable characteristics in his way of playing. We will use the ML Agents framework, which allows to simulate games and train directly from them and generate demos for imitation learning. However, we will also discuss how can we train from external datasets, and which data they should have.

The dataset used as input for the neural network is formed by in-game simplified frames and the key/mouse inputs made in that moment (mouse movement and keys pressed). Using that dataset, a neural network will be trained to mimic that NPC by receiving simplified game frames as input, with the objective of obtaining a neural network that visually reproduces that NPC’s way of playing.


\section{Work Motivation}

This topic was chosen because I found interesting the potential of neural networks in solving difficult problems and how well they solve them. Also, I wanted to learn to use neural networks and make them.

On the other hand, one of the main motivations of this work was to conduct a research article (in parallel, with the Study and Research at the UJI program). I found interesting that almost every scientific work related to neural networks was oriented to learn to play optimally specific games, but almost none had the objective of imitating real players in that games ~\cite{livingstone}~\cite{nakano}, and that motivated me to investigate deeply in that area.

\section{Objectives}

The main objectives are the following:
\begin{itemize}
 \item Program a simple shooting game using Unity3D.
 \item Obtain in-game information from Unity3D to train an agent
 \item Obtain a trained neural network that can reproduce the movements and reactions of one specific player.
 \item Develop and define a framework that allows to imitate real human players in more complex video games having their games’ data (video games where you can walk or move in many other ways, games with more complex graphics or a larger amount of controls).
\end{itemize}

\section{Environment and Initial State}
\label{sec:initialstate}

This project was intended to be developed with one PC, and trained at the research laboratory of my supervisor in this TFG to speed up the training process. However, circumstances due to COVID-19 made impossible using that equipment, and therefore delaying some steps of the project.
